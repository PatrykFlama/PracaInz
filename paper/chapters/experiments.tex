Przez eksperymenty chcieliśmy pokazać wyraźnie szybsze działanie algorytmu ze skokami w stosunku do reszty. Chcieliśmy również zbadać zależność zmiany parametrów liczby próbek oraz długości próbek w stosunku do czasu wykonywania. W poniższych wynikach zdecydowaliśmy się pominąć przedstawienie heurystyki, ze względu na jej zbyt długi czas działania w stosunku do reszty.

Testy przeprowadzaliśmy na stałych parametrach, jedynie badany parametr jest zmieniany. Automat oraz próbki przygotowywaliśmy w podany sposób:
- generowaliśmy losowy automat pełny
- na podstawie tak wygenerowanego automatu generowaliśmy losowe próbki
- losowo usuwaliśmy przejścia z automat, które są zawarte w próbce/próbkach.
Dzięki temu automatycznie wykluczamy wszystkie przypadki przejść, które mogłyby zostać poprowadzone w dowolny sposób i zakłócać testowanie.