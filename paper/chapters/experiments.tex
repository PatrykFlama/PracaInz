\section{Środowisko}
Testy zostały wykonane na środowisku o poniższych parametrach:
\begin{itemize}
  \item System:
  \item Procesor:
  \item Architektura:
  \item Szybkość procesora: 
\end{itemize}

\section{Cel}
Przez eksperymenty chcieliśmy pokazać wyraźnie szybsze działanie algorytmu ze skokami w stosunku do reszty. Chcieliśmy również zbadać zależność zmiany parametrów liczby próbek oraz długości próbek w stosunku do czasu wykonywania. W poniższych wynikach zdecydowaliśmy się pominąć przedstawienie heurystyki, ze względu na jej zbyt długi czas działania w stosunku do reszty.

\section{Sposób testowania}
Testy przeprowadzaliśmy na stałych parametrach, jedynie badany parametr jest zmieniany. Automat oraz próbki przygotowywaliśmy w podany sposób:
\begin{itemize}
  \item generowaliśmy losowy automat pełny,
  \item na podstawie tak wygenerowanego automatu generowaliśmy losowe próbki,
  \item losowo usuwaliśmy przejścia z automatu, które są zawarte w próbce lub próbkach.
\end{itemize}
Dzięki temu automatycznie wykluczamy wszystkie przypadki przejść w automacie, które mogłyby zostać poprowadzone w dowolny sposób i zakłócać testowanie.

Przy rysowaniu wykresów średni czas wykonania liczyliśmy na podstawie okna przesuwnego, tak żeby zredukować szum.

\section{Wyniki}

\subsection{Test 1}
Pierwszy test Rys.~\ref{fig:sample_length} został wykonany na parametrach: 
\begin{itemize}
  \item liczba próbek z zakresu [10,1000], 
  \item liczba stanów: 20,
  \item liczba próbek: 30,
  \item rozmiar alfabetu: 5,
  \item brakujące krawędzie: 4,
  \item wariancja długości próbek: 0,2.
\end{itemize}

Dla obu algorytmów czas rośnie wraz z długością próbek. Analiza wykresu pokazuje, że algorytm ze skokami jest znacznie szybszy niż algorytm naiwny.


\subsection{Test 2}
Drugi test Rys.~\ref{fig:num_samples} został wykonany na parametrach: 
\begin{itemize}
  \item liczba próbek: 30, 
  \item liczba stanów: 20,
  \item liczba próbek z zakresu [30,1000],
  \item rozmiar alfabetu: 5,
  \item brakujące krawędzie: 4,
  \item wariancja długości próbek: 0,2.
\end{itemize}

Dla algorytmu ze skokami czas wykonania rośnie wraz z liczbą próbek, natomiast algorytm naiwny wykazuje skoki czasowe. W porównaniu z wcześniejszym testem tempo wzrostu jest wyraźnie niższe.

\begin{figure} 
  \centering
  \includegraphics[width=0.8\textwidth]{charts/sample_length.pdf}
  \caption{Średni czas wykonania algorytmu w zależności od długości próbek, liczony na oknie przesuwnym 500.}
  \label{fig:sample_length}
\end{figure}

\begin{figure}
  \centering
  \includegraphics[width=0.8\textwidth]{charts/num_samples.pdf}
  \caption{Średni czas wykonania algorytmu w zależności od liczby próbek, liczony na oknie przesuwnym 500.}
  \label{fig:num_samples}
\end{figure}