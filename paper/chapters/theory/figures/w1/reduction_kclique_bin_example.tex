\begin{figure}[h]
\centering
\begin{tikzpicture}[
    scale=0.95,
    ->,
    >=Stealth,
    node distance=14mm,
    state/.style={circle, draw, minimum size=8mm},
    small/.style={circle, draw, minimum size=5mm},
    accept/.style={circle, draw, double, minimum size=8mm},
    reject/.style={circle, draw, minimum size=8mm},
    tostart/.style={state, dashed, thick},
    every label/.style={font=\small}
]

    % Left section
    \node[draw,state,inner sep=2pt] (v1) at (0,0) {$s_1$};
    \node[draw,state,inner sep=2pt] (v2) at (3,1) {$s_2$};
    \node[draw,state,inner sep=2pt] (v3) at (3,-1) {$s_3$};

    \draw [->,thick] (v1) -- (v2) node[midway,above] {$bin(a)$};
    \draw [->,thick] (v1) -- (v3) node[midway,below] {$bin(b)$};

    % equiv arrow
    \node at (4.5,0) {$\Leftrightarrow$};

    % Right section
    %% States
    \node[draw,state,inner sep=2pt] (b1) at (6,0) {$s_1$};
    \node[draw,state,inner sep=2pt] (b2) at (8,0) {$ $};
    \node[draw,state,inner sep=2pt] (b3) at (10,0) {$ $};
    \node[draw,state,inner sep=2pt] (b4) at (12,1) {$s_2$};
    \node[draw,state,inner sep=2pt] (b5) at (12,-1) {$s_3$};
    %% Edges
    \draw [->,thick] (b1) -- (b2) node[midway,above] {$1$};
    \draw [->,thick] (b2) -- (b3) node[midway,above] {$0$};
    \draw [->,thick] (b3) -- (b4) node[midway,above] {$1$};
    \draw [->,thick] (b3) -- (b5) node[midway,below] {$0$};

\end{tikzpicture}
\caption{Przykład reprezentacji krawędzi z~etykietą $bin$.}
\label{fig:kclique-bin-example}
\end{figure}