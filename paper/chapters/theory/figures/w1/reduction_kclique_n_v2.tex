\begin{figure}[h]
\centering
\begin{tikzpicture}[
        ->,
    >=Stealth,
    node distance=14mm,
    state/.style={circle, draw, minimum size=8mm},
    small/.style={circle, draw, minimum size=5mm},
    accept/.style={circle, draw, double, minimum size=8mm},
    reject/.style={circle, draw, minimum size=8mm},
    tostart/.style={state, dashed, thick},
    every label/.style={font=\small}
]
    % wierzchołek startowy
    \node[draw,state,inner sep=2pt] (s) at (0,-2) {$q_{\lambda}$};

    % wierzchołek sink
    \node[draw,accept,inner sep=2pt] (sinkp) at (0,-4) {$\text{sink}_+$};
    \draw[->,thick] (s) -- (sinkp) node[midway,above] {$c_1,c_2$};

    \node[draw,reject,inner sep=2pt] (sinkn) at (1.5,-4) {$\text{sink}_-$};

    % punkty docelowe dla krawędzi
    \coordinate (t1) at (-1.5,0);
    \coordinate (tx) at (0,0);
    \coordinate (tlast) at (1.5,0);

    % krawędzie p
    \draw[->,dashed,thick] (s) -- (t1) node[midway,left] {$p_{1}$};
    \node at (-0.6, -0.2) {$\dots$};
    \draw[->,dashed,thick] (s) -- (tx) node[midway,left] {$p_{x}$};
    \node at (0.7, -0.2) {$\dots$};
    \draw[->,dashed,thick] (s) -- (tlast) node[midway,left] {$p_{k}$};

    % wierzchołki v
    \node[draw,state,inner sep=1.5pt] (v1) at (3,-0.4) {$v_{1}$};
    \node (vdots1) at (3,-1.1) {$\vdots$};
    \node[draw,state,inner sep=1.5pt] (vi) at (3,-2) {$v_{i}$};
    \node (vdots2) at (3,-2.7) {$\vdots$};
    \node[draw,state,inner sep=1.5pt] (vn) at (3,-3.6) {$v_{n}$};

    % krawędź test
    \node[draw,state,inner sep=2pt] (test_edge) at (5,-2) {$ $};
    \draw[->,thick] (vi) -- (test_edge) node[midway,above] {$t$};

    % sekcja weryfikacji wierzchołka
    \node[draw,tostart,inner sep=2pt] (vv1) at (7,-0.4) {$q_{\lambda}$};
    \node (vdots1) at (7,-1.1) {$\vdots$};
    \node[draw,state,inner sep=2pt] (vvi) at (7,-2) {$u_i$};
    \node (vdots1) at (7,-2.7) {$\vdots$};
    \node[draw,tostart,inner sep=2pt] (vvn) at (7,-3.6) {$q_{\lambda}$};

    \draw[->,thick] (test_edge) -- (vv1) node[midway,above] {$e_1$};
    \draw[->,thick] (test_edge) -- (vvi) node[midway,above] {$e_i$};
    \draw[->,thick] (test_edge) -- (vvn) node[midway,above] {$e_n$};

    % sekcja wyboru
    \node[draw,tostart,inner sep=2pt] (choice1) at (9,-1.1) {$q_{\lambda}$};
    \node[draw,state,inner sep=2pt] (choice2) at (9,-2.7) {$ $};
    \draw[->,thick] (vvi) -- (choice1) node[midway,above] {$c_1$};
    \draw[->,thick] (vvi) -- (choice2) node[midway,below] {$c_2$};

    % sekcja testu sąsiedztwa
    \node[accept, inner sep=1.5pt] (accept) at (11,-1.1) {};
    \node[reject, inner sep=1.5pt] (reject) at (11,-4.3) {};
    \draw[->,thick] (choice2) -- (accept) node[midway,right] {$e_j \ \text{s.t.} \ (i, j) \in E$};
    \draw[->,thick] (choice2) -- (reject) node[midway,right] {$e_j \ \text{s.t.} \ (i, j) \notin E$};

    \coordinate (label_end) at (13,-4.5);
    % zielona ramka
    \node[
        draw=green!60!black,
        thick,
        rounded corners,
        fit=(vv1)(vvi)(vvn)(choice1)(choice2)(accept)(reject)(test_edge)(label_end),
        label={[green!60!black]above:dla każdego $v_i$}
    ] {};

\end{tikzpicture}
\caption{Konstrukcja automatu dla redukcji z k-klik, korzystająca z alfabetu o rozmiarze zależnym od liczby wierzchołków grafu}
\label{fig:kcliquen-v2}
\end{figure}