\begin{figure}[h]
\centering
\begin{tikzpicture}[
    scale=0.8,
    ->,
    >=Stealth,
    node distance=14mm,
    state/.style={circle, draw, minimum size=8mm},
    small/.style={circle, draw, minimum size=5mm},
    accept/.style={circle, draw, double, minimum size=8mm},
    reject/.style={circle, draw, minimum size=8mm},
    every label/.style={font=\small}
]
    % wierzchołek startowy
    \node[draw,state,inner sep=2pt] (s) at (0,0) {$q_{\lambda}$};

    % wierzchołek sink
    \node[draw,accept,inner sep=2pt] (s1) at (0,-2) {$ $};
    \draw[->,thick] (s) -- (s1) node[midway,right] {$2*1$};
    \node[draw,accept,inner sep=1.5pt] (s2) at (0,-4) {$ $};
    \draw[->,thick] (s1) -- (s2) node[midway,right] {$16*1$};


    % wierzchołki wyliczajce indeks v
    \node[draw,state,inner sep=2pt] (p1) at (2,0) {$p_1$};
    \draw[->,thick] (s) -- (p1) node[midway,above] {$0$};
    \node[draw,state,inner sep=2pt] (p2) at (4,0) {$p_2$};
    \draw[->,thick] (p1) -- (p2) node[midway,above] {$0$};
    \node (pdots) at (6,0) {$\dots$};
    \draw[->,thick] (p2) -- (pdots) node[midway,above] {$0$};
    \node[draw,state,inner sep=2pt] (pk) at (8,0) {$p_k$};
    \draw[->,thick] (pdots) -- (pk) node[midway,above] {$0$};

    % przejścia wyboru indeksu v
    \coordinate (t1) at (2,-1.5);
    \draw[->,dashed,thick] (p1) -- (t1) node[midway,right] {$1$};
    \coordinate (t2) at (4,-1.5);
    \draw[->,dashed,thick] (p2) -- (t2) node[midway,right] {$1$};
    \coordinate (tk) at (8,-1.5);
    \draw[->,dashed,thick] (pk) -- (tk) node[midway,right] {$1$};

    % wierzchołki v
    \node[draw,state,inner sep=1.5pt] (v1) at (2,-3) {$v_{1}$};
    \node (vdots1) at (3,-3) {$\dots$};
    \node[draw,state,inner sep=1.5pt] (vi) at (4,-3) {$v_{i}$};
    \node (vdots2) at (5,-3) {$\dots$};
    \node[draw,state,inner sep=1.5pt] (vn) at (6,-3) {$v_{n}$};


    % pierwsza warstwa (check 1)
    \node[draw,state,inner sep=2pt] (u0) at (4,-5) {$ $};
    \node[draw,state,inner sep=2pt] (u1) at (6,-5) {$u_1$};
    \node (udots1) at (8,-5) {$\dots$};
    \node[draw,state,inner sep=2pt] (ui) at (10,-5) {$u_i$};
    \node (udots2) at (12,-5) {$\dots$};
    \node[draw,state,inner sep=2pt] (un) at (14,-5) {$u_n$};

    \draw[->,thick] (vi) -- (u0) node[midway,right] {$10*1$};
    \draw[->,thick] (u0) -- (u1) node[midway,above] {$0$};
    \draw[->,thick] (u1) -- (udots1) node[midway,above] {$0$};
    \draw[->,thick] (udots1) -- (ui) node[midway,above] {$0$};
    \draw[->,thick] (ui) -- (udots2) node[midway,above] {$0$};
    \draw[->,thick] (udots2) -- (un) node[midway,above] {$0$};

    \node[draw,state,inner sep=2pt] (u1s) at (6,-7) {$q_{\lambda}$};
    \node[draw,state,inner sep=2pt] (uis) at (10,-7) {$q_{\lambda}$};
    \node[draw,state,inner sep=2pt] (uns) at (14,-7) {$q_{\lambda}$};

    \draw[->,thick] (u1) -- (u1s) node[midway,right] {$1$};
    \draw[->,thick] (ui) -- (uis) node[midway,right] {$2*1$};
    \draw[->,thick] (un) -- (uns) node[midway,right] {$1$};

    % druga warstwa (check 2)
    \node[draw,state,inner sep=2pt] (w0) at (4,-9) {$ $};
    \node[draw,state,inner sep=2pt] (w1) at (6,-9) {$w_1$};
    \node (wdots1) at (8,-9) {$\dots$};
    \node[draw,state,inner sep=2pt] (wi) at (10,-9) {$w_i$};
    \node (wdots2) at (12,-9) {$\dots$};
    \node[draw,state,inner sep=2pt] (wn) at (14,-9) {$w_n$};

    \draw[->,thick] (u0) -- (w0) node[midway,right] {$1$};
    \draw[->,thick] (w0) -- (w1) node[midway,above] {$0$};
    \draw[->,thick] (w1) -- (wdots1) node[midway,above] {$0$};
    \draw[->,thick] (wdots1) -- (wi) node[midway,above] {$0$};
    \draw[->,thick] (wi) -- (wdots2) node[midway,above] {$0$};
    \draw[->,thick] (wdots2) -- (wn) node[midway,above] {$0$};

    \node[draw,state,inner sep=2pt] (w1s) at (6,-11) {$q_{\lambda}$};
    \node[draw,state,inner sep=2pt] (wns) at (14,-11) {$q_{\lambda}$};

    \draw[->,thick] (w1) -- (w1s) node[midway,right] {$3*1$};
    \draw[->,thick] (wn) -- (wns) node[midway,right] {$3*1$};

    % warstwa sąsiedztwa
    \node[draw,state,inner sep=2pt] (neigh1) at (10,-13) {$n_1$};
    \node (neighdots1) at (12,-13) {$\dots$};
    \node[draw,state,inner sep=2pt] (neighi) at (14,-13) {$n_j$};
    \node (neighdots2) at (16,-13) {$\dots$};
    \node[draw,state,inner sep=2pt] (neighn) at (18,-13) {$n_n$};

    \draw[->,thick] (wi) -- (neigh1) node[midway,right] {$4*1$};
    \draw[->,thick] (neigh1) -- (neighdots1) node[midway,above] {$0$};
    \draw[->,thick] (neighdots1) -- (neighi) node[midway,above] {$0$};
    \draw[->,thick] (neighi) -- (neighdots2) node[midway,above] {$0$};
    \draw[->,thick] (neighdots2) -- (neighn) node[midway,above] {$0$};

    \node[accept, inner sep=1.5pt] (accept) at (13,-15) {};
    \node[reject, inner sep=1.5pt] (reject) at (15,-15) {};
    \draw[->,thick] (neighi) -- (accept) node[midway,left] {$v_j \in N(v_i)$};
    \draw[->,thick] (neighi) -- (reject) node[midway,right] {$v_j \notin N(v_i)$};

% TODO: this labeling of leading to neigh state sould be more precise

    % zielona ramka
    \node[
        draw=green!60!black,
        thick,
        rounded corners,
        fit=(u0)(neighn)(accept)(reject),
        label={[green!60!black]above:dla każdego $v_i$}
    ] {};

\end{tikzpicture}
\caption{Konstrukcja automatu dla redukcji z k-klik, korzystająca z alfabetu o rozmiarze 2}
\label{fig:kclique2}
\end{figure}