\begin{figure}[h]
\centering
\begin{tikzpicture}[
        ->,
    >=Stealth,
    node distance=14mm,
    state/.style={circle, draw, minimum size=8mm},
    small/.style={circle, draw, minimum size=5mm},
    accept/.style={circle, draw, double, minimum size=8mm},
    reject/.style={circle, draw, minimum size=8mm},
    every label/.style={font=\small}
]
    % wierzchołek startowy
    \node[draw,state,inner sep=2pt] (s) at (0,-2) {$q_{\lambda}$};

    % wierzchołek sink
    \node[draw,accept,inner sep=2pt] (sink) at (0,-4) {sink};
    \draw[->,thick] (s) -- (sink) node[midway,above] {$\text{choice}_1,\text{choice}_2$};

    % wierzchołki v
    \node[draw,state,inner sep=1.5pt] (v1) at (4,0) {$v_{1}$};
    \node (vdots1) at (4,-0.7) {$\vdots$};
    \node[draw,state,inner sep=1.5pt] (vi) at (4,-1.6) {$v_{i}$};
    \node[draw,state,inner sep=1.5pt] (vj) at (4,-2.6) {$v_{j}$};
    \node (vdots2) at (4,-3.2) {$\vdots$};
    \node[draw,state,inner sep=1.5pt] (vn) at (4,-4.0) {$v_{n}$};

    % punkty docelowe dla krawędzi
    \coordinate (t1) at (3,-0.8);
    \coordinate (tx) at (3,-1.6);
    \coordinate (ty) at (3,-2.4);
    \coordinate (tlast) at (3,-3.2);

    % krawędzie p
    \draw[->,dashed,thick] (s) -- (t1) node[midway,above] {$p_{1}$};
    \node at (2.8,-1.2) {$\vdots$};
    \draw[->,dashed,thick] (s) -- (tx) node[midway,above] {$p_{x}$};
    \draw[->,dashed,thick] (s) -- (ty) node[midway,above] {$p_{y}$};
    \node at (2.8,-2.6) {$\vdots$};
    \draw[->,dashed,thick] (s) -- (tlast) node[midway,above] {$p_{k}$};

    % sekcja weryfikacji wierzchołka
    \node[draw,state,inner sep=2pt] (vv1) at (5.5,-0.2) {$q_{\lambda}$};
    \node (vdots1) at (5.5,-0.8) {$\vdots$};
    \node[draw,state,inner sep=2pt] (vvi) at (5.5,-1.6) {$u_i$};
    \node (vdots1) at (5.5,-2.2) {$\vdots$};
    \node[draw,state,inner sep=2pt] (vvn) at (5.5,-3) {$q_{\lambda}$};

    \draw[->,thick] (vi) -- (vv1) node[midway,above] {$e_1$};
    \draw[->,thick] (vi) -- (vvi) node[midway,above] {$e_i$};
    \draw[->,thick] (vi) -- (vvn) node[midway,above] {$e_n$};

    % sekcja wyboru
    \node[draw,state,inner sep=2pt] (choice1) at (8,-0.8) {$q_{\lambda}$};
    \node[draw,state,inner sep=2pt] (choice2) at (8,-2.4) {$c_2$};
    \draw[->,thick] (vvi) -- (choice1) node[midway,above] {$\text{choice}_1$};
    \draw[->,thick] (vvi) -- (choice2) node[midway,below] {$\text{choice}_2$};

    % sekcja testu sąsiedztwa
    \node[accept, inner sep=1.5pt] (accept) at (11,-1.4) {};
    \node[reject, inner sep=1.5pt] (reject) at (11,-3) {};
    \draw[->,thick] (choice2) -- (accept) node[midway,above] {$e_j | v_j \in N(v_i)$};
    \draw[->,thick] (choice2) -- (reject) node[midway,below] {$e_j | v_j \notin N(v_i)$};

    % zielona ramka
    \node[
        draw=green!60!black,
        thick,
        rounded corners,
        fit=(vv1)(vvi)(vvn)(choice1)(choice2)(accept)(reject),
        label={[green!60!black]above:dla każdego $v_i$}
    ] {};

\end{tikzpicture}
\caption{Konstrukcja automatu dla redukcji z k-klik, korzystająca z alfabetu o rozmiarze zależnym od liczby wierzchołków grafu}
\label{fig:kcliquen}
\end{figure}