\begin{figure}[h]
\centering
\begin{tikzpicture}[
    scale=0.8,
    ->,
    >=Stealth,
    node distance=14mm,
    state/.style={circle, draw, minimum size=8mm},
    small/.style={circle, draw, minimum size=5mm},
    accept/.style={circle, draw, double, minimum size=8mm},
    reject/.style={circle, draw, minimum size=8mm},
    tostart/.style={state, dashed, thick},
    every label/.style={font=\small}
]
    % start
    \node[draw,state,inner sep=2pt] (s) at (0,0) {$q_{\lambda}$};

    % sinki
    \node[draw,reject,inner sep=1.5pt] (sinkn) at (0,-3) {$\text{sink}_-$};

    % wyliczanie krawędzi p
    \node[draw,state,inner sep=2pt] (p1) at (2,0) {$p_1$};
    \draw[->,thick] (s) -- (p1) node[midway,above] {$0$};
    \node[draw,state,inner sep=2pt] (p2) at (4,0) {$p_2$};
    \draw[->,thick] (p1) -- (p2) node[midway,above] {$0$};
    \node (pdots) at (6,0) {$\dots$};
    \draw[->,thick] (p2) -- (pdots) node[midway,above] {$0$};
    \node[draw,state,inner sep=2pt] (pk) at (8,0) {$p_k$};
    \draw[->,thick] (pdots) -- (pk) node[midway,above] {$0$};

    % przejścia wyboru indeksu v
    \coordinate (t1) at (2,-1.5);
    \draw[->,dashed,thick] (p1) -- (t1) node[midway,right] {$1$};
    \coordinate (t2) at (4,-1.5);
    \draw[->,dashed,thick] (p2) -- (t2) node[midway,right] {$1$};
    \coordinate (tk) at (8,-1.5);
    \draw[->,dashed,thick] (pk) -- (tk) node[midway,right] {$1$};

    % wierzchołki v
    \node[draw,state,inner sep=1.5pt] (v1) at (2,-3) {$s_{1}$};
    \node (vdots1) at (3,-3) {$\dots$};
    \node[draw,state,inner sep=1.5pt] (vi) at (4,-3) {$s_{i}$};
    \node (vdots2) at (5,-3) {$\dots$};
    \node[draw,state,inner sep=1.5pt] (vn) at (6,-3) {$s_{n}$};

    % krawędź testowa 
    \node[draw,state,inner sep=2pt] (test_edge) at (4,-5) {$ $};
    \draw[->,thick] (vi) -- (test_edge) node[midway,right] {$10*1$};

    % wyliczanie krawędzi e_j
    \node[draw,state,inner sep=2pt] (u1) at (6,-5) {$e_1$};
    \node (udots1) at (8,-5) {$\dots$};
    \node[draw,state,inner sep=2pt] (uj) at (10,-5) {$e_j$};
    \node (udots2) at (12,-5) {$\dots$};
    \node[draw,state,inner sep=2pt] (un) at (14,-5) {$e_n$};

    \draw[->,thick] (test_edge) -- (u1) node[midway,above] {$0$};
    \draw[->,thick] (u1) -- (udots1) node[midway,above] {$0$};
    \draw[->,thick] (udots1) -- (uj) node[midway,above] {$0$};
    \draw[->,thick] (uj) -- (udots2) node[midway,above] {$0$};
    \draw[->,thick] (udots2) -- (un) node[midway,above] {$0$};

    % sprawdzanie czy element jest w zbiorze
    \node[draw,accept,inner sep=2pt] (inset) at (9,-7) {$\text{sink}_+$};
    \node[draw,tostart,inner sep=2pt] (outset) at (11,-7) {$q_{\lambda}$};

    \draw[->,thick] (uj) -- (inset) node[midway,left] {$j \in S_i$};
    \draw[->,thick] (uj) -- (outset) node[midway,right] {$j \notin S_i$};

    % ramka zielona: gadget per s_i
    \node[
        draw=green!60!black,
        thick,
        rounded corners,
        fit=(test_edge)(un)(inset)(outset),
        label={[green!60!black]above:dla każdego $s_i$}
    ] {};

\end{tikzpicture}
\caption{Konstrukcja automatu dla redukcji z $k$-Set Cover, korzystająca z alfabetu o rozmiarze 2}
\label{fig:ksetcover2}
\end{figure}
