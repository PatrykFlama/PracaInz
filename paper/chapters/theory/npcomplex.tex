Aby wykazać że problem naprawiania częściowego DFA \ref{def:fixpartialdfa} jest $\mathbb{NP}$-zupełny, musimy udowodnić, że należy do klasy $\mathbb{NP}$ oraz że jest $\mathbb{NP}$-trudny.


\begin{lemma}
[Problem naprawienia częściowego DFA należy do klasy $\mathbb{NP}$] 
\end{lemma}

\vspace{-0.8cm}

\begin{proof}
Problem naprawienia częściowego DFA należy do klasy $\mathbb{NP}$, ponieważ dla danej instancji problemu możemy w czasie wielomianowym zweryfikować poprawność podanego rozwiązania. \\

Weryfikacja polega na sprawdzeniu, czy uzupełniony automat jest deterministyczny oraz czy akceptuje i odrzuca odpowiednie próbki. Sprawdzenie deterministyczności automatu wymaga przejrzenia wszystkich stanów i liter alfabetu, co zajmuje czas $O(|Q| \cdot |\Sigma|)$. Sprawdzenie klasyfikacji próbek wymaga przejścia przez każdą próbkę i symulacji jej działania na automacie, co zajmuje czas $O((|S^+| + |S^-|) \cdot M)$, gdzie $M$ to maksymalna długość próbki. Ponieważ oba te kroki można wykonać w czasie wielomianowym względem rozmiaru wejścia, problem naprawienia częściowego DFA należy do klasy $\mathbb{NP}$.
\end{proof}


\begin{fact}
Problem \textit{najmniejszego zgodnego automatu} jest $\mathbb{NP}$-zupełny \cite{grammatical_inference}:
\end{fact}

\begin{definition}[Problem najmniejszego zgodnego automatu]   
% Otrzymujemy liczbę natrualną $n \in \mathbb{N}$ oraz dwa zbiory słów nad alfabetem $\Sigma$: zbiór słów akceptowanych $S^+ \subseteq \Sigma^{*}$ oraz zbiór słów odrzucanych $S^{-} \subseteq \Sigma^{*}$. 
% Należy odpowiedzieć na pytanie, czy istnieje deterministyczny automat skończony (DFA) $\mathcal{A}$, z co najwyżej $n$ stanami, taki że wszystkie słowa z $S^+$ są akceptowane przez $\mathcal{A}$, a wszystkie słowa z $S^-$ są odrzucane przez $\mathcal{A}$.

\textbf{Wejście:} liczba naturalna $n \in \mathbb{N}$ oraz dwa zbiory słów nad alfabetem $\Sigma$: zbiór słów akceptowanych $S^+ \subseteq \Sigma^{*}$ oraz zbiór słów odrzucanych $S^{-} \subseteq \Sigma^{*}$.

\vspace{0.5cm}

\textbf{Wyjście:} odpowiedź, czy istnieje deterministyczny automat skończony (DFA) $\mathcal{A}$, z co najwyżej $n$ stanami, taki że wszystkie słowa z $S^+$ są akceptowane przez $\mathcal{A}$, a wszystkie słowa z $S^-$ są odrzucane przez $\mathcal{A}$.

\label{def:minimal_consistent_dfa}
\end{definition}


\begin{lemma}
[Problem naprawienia częściowego DFA jest $\mathbb{NP}$-trudny]
\end{lemma}

\vspace{-0.8cm}

\begin{proof}
    Możemy przeprowadzić redukcję z problemu \textit{najmniejszego zgodnego automatu} do naszego problemu \textit{naprawienia częściowego DFA}. \\
    Weźmy instancję problemu \textit{najmniejszego zgodnego automatu} - liczbę naturalną $n$ oraz zbiory słów $S^+$ oraz $S^-$. Stwórzmy częściowy automat DFA $A = (Q, \Sigma, \delta, q_\lambda, \mathbb{F_A}, \mathbb{F_R})$, gdzie $|Q| = n$ oraz
    dla wszystkich stanów $q \in Q$ oraz liter $a \in \Sigma$ przejście $\delta(q, a)$ jest nieokreślone,
    natomiast zbiory próbek są takie same jak w oryginalnym problemie.\\
    Wtedy odpowiedź na problem \textit{naprawienia częściowego DFA} dla automatu $A$ oraz próbek $S^+$ i $S^-$ jest pozytywna wtedy i tylko wtedy gdy odpowiedź na problem \textit{najmniejszego zgodnego automatu} dla liczby $n$ oraz próbek $S^+$ i $S^-$ jest pozytywna. \\
\end{proof}
