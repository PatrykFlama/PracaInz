W tym podrozdziale pokażemy, że problem naprawienia $k$-częściowego DFA należy do klasy złożoności parametrycznej $W[P]$. W tym celu skonstruujemy $\kappa$-ograniczoną niedeterministyczną maszynę Turinga, która w ograniczonej liczbie kroków niedeterministycznych zgaduje brakujące przejścia automatu, a następnie w czasie wielomianowym weryfikuje poprawność otrzymanego uzupełnienia względem zadanych przykładów pozytywnych i~negatywnych.

\begin{definition}[Zbiór $\mathbb{N}_0\lbrack X \rbrack$]
    Przez $\mathbb{N}_0[X]$ oznaczamy zbiór wielomianów w zmiennej $X$
    o współczynnikach z~nieujemnych liczb naturalnych, tj.
    \[
    \mathbb{N}_0[X]
    =
    \{\, p(X) = \sum_{i=0}^d a_i X^i \mid d \in \mathbb{N},\; a_i \in \mathbb{N}_0 \,\}.
    \]
\end{definition}

\begin{definition}[$\kappa$-ograniczona niedeterministyczna maszyna Turinga]
    Niech $\Sigma$ będzie alfabetem, a $\kappa : \Sigma^* \to \mathbb{N}$ niech będzie parametryzacją. Niedeterministyczna maszyna Turinga $M$ z~alfabetem wejściowym $\Sigma$ nazywana jest \emph{$\kappa$-ograniczoną}, jeśli istnieją obliczalne funkcje $f, h : \mathbb{N} \to \mathbb{N}$ oraz wielomian $p \in \mathbb{N}_0[X]$ taki, że przy każdym przebiegu z~wejściem $x \in \Sigma^*$ maszyna $M$ wykonuje co najwyżej $f(k) \cdot p(n)$ kroków, z czego co najwyżej $h(k) \cdot \log n$ kroków jest niedeterministycznych.  
    Tutaj $n := |x|$, a $k := \kappa(x)$.
\end{definition}

\begin{definition}[Klasa $W \lbrack P \rbrack$]
$W[P]$ jest klasą wszystkich problemów parametryzowanych $(Q, \kappa)$, które mogą być rozwiązane przez $\kappa$-ograniczoną niedeterministyczną maszynę Turinga.
\end{definition}
  

\begin{theorem}
    Problem naprawienia $k$-częściowego DFA należy do $W[P]$.
\end{theorem} 

\begin{proof}
    Aby pokazać, że problem naprawienia częściowego DFA należy do klasy $W[P]$, konstruujemy niedeterministyczną maszynę Turinga działającą w następujący sposób: 

    Niech $A = (Q, \Sigma, \delta, q_\lambda, \mathbb{F_A}, \mathbb{F_R})$ będzie częściowym automatem deterministycznym, a~$S^+ \subseteq \Sigma^*$ i~$S^- \subseteq \Sigma^*$ zbiorem słów pozytywnych i~negatywnych. Oznaczmy przez $k$ liczbę brakujących przejść w $\delta$.  

    Maszyna Turinga zgaduje $k$ stanów docelowych wybrakowanych przejść, w $O(k \cdot \log |Q|)$ krokach niedeterministycznych. Następnie deterministycznie weryfikuje, czy zgadnięte uzupełnienie powoduje akceptację wszystkich słów z~$S^+$ i~odrzucenie wszystkich słów z~$S^-$. Tę weryfikację wykonuje w czasie wielomianowym względem rozmiaru wejścia - dla przykładu symulując działanie automatu.

    Cała niedeterministyczna maszyna Turinga wykonuje liczbę kroków niedeterministycznych zależną tylko od parametru $k$~i~sprawdza poprawność w czasie wielomianowym względem rozmiaru wejścia. Stąd problem naprawienia częściowego DFA należy do klasy $W[P]$.

\end{proof}
