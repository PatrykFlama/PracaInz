
\begin{definition}[Problem k-klik]
% \noindent Problem k-klik to zparametryzowana wersja problemu kliki, w której dodatkowo podana jest liczba całkowita $k$ i należy odpowiedzieć, czy w grafie istnieje klika o rozmiarze co najmniej $k$.

% \noindent\textbf{Wejście:} graf nieskierowany $G = (V, E)$ oraz liczba całkowita $k$.

% \noindent \textbf{Wyjście:} odpowiedź, czy w grafie $G$ istnieje kliką o rozmiarze co najmniej $k$, czyli podzbiór $V' \subseteq V$ taki, że $|V'| \geq k$ oraz dla każdej pary wierzchołków $u, v \in V'$ zachodzi $(u, v) \in E$.

Weźmy graf nieskierowany $G = (V, E)$ oraz liczbę całkowita $k$. Problem k-klik polega na znalezieniu w grafie $G$ kliki o rozmiarze co najmniej $k$, czyli podzbioru $V' \subseteq V$ takiego, że $|V'| \geq k$ oraz dla każdej pary wierzchołków $u, v \in V'$ zachodzi $(u, v) \in E$.

\noindent Problem k-klik należy do klasy W[1]-zupełnych problemów \cite{parametrized_algorithms}.
\end{definition}

Jeżeli w grafie wejściowym istnieje klika rozmiaru większego od $k$, to zawiera ona w sobie klikę rozmiaru dokładnie $k$, więc możemy rozważać wersję problemu k-klik, w której szukamy kliki o rozmiarze dokładnie $k$.

Pokażemy, jak dla dowolnego wejścia do problemu k-klik skonstruować automat oraz próbki, będące wejściem do problemu naprawienia częściowego DFA, tak aby rozwiązanie problemu naprawienia częściowego DFA istniało wtedy i tylko wtedy, gdy w grafie istnieje klika o rozmiarze co najmniej $k$. Dodatkowo pokażemy jak z rozwiązanego problemu naprawienia częściowego DFA wyprowadzić rozwiązanie problemu k-klik.

\begin{theorem}
Problem naprawienia częściowego DFA jest W[1]-trudny względem parametru liczby stanów $|Q|$.
\end{theorem}

W poniższych rozważaniach przyjmiemy konwencję, w której $V = \{V_1, V_2, .., V_n\}$ to zbiór wierzchołków o rozmiarze $n = |V|$ grafu wejściowego, a $E$ to zbiór krawędzi grafu wejściowego. \\
Stany automatu będą odpowiadały wierzchołkom grafu wejściowego - dla każdego wierzchołka $V_i \in V$ stworzymy stan $v_i \in Q$. \\
Dodatkowo ciąg liter z alfabetu $\Sigma$ zapisany w nawiasach kwadratowych $[a \ b \ c]$ oznacza próbkę składającą się z liter $a$, $b$ oraz $c$. 

\subsubsection{Redukcja dla rozmiaru alfabetu zależnego od rozmiaru grafu}

Na rysunku \ref{fig:kcliquen-v2} przedstawiona jest konstrukcja automatu dla redukcji z k-klik, korzystająca z alfabetu o rozmiarze zależnym od liczby wierzchołków grafu. \\
Każdy stan $q_{\lambda}$ odpowiada temu samemu stanowi początkowemu automatu.
Przejścia $p_i$ oznaczone przerywaną linią odpowiadają brakującym przejściom w automacie. Każde przejście, które nie jest zaznaczone i nie wychodzi ze stanu $\text{sink}_+$ lub $\text{sink}_-$, prowadzi do stanu odrzucającego $\text{sink}_-$. Niezaznaczone przejścia wychodzące ze stanów $\text{sink}_+$ i $\text{sink}_-$ prowadzą do tych samych stanów. \\
Część automatu zaznaczona ramką jest powielana dla każdego stanu $v_.$. \\

\import{theory/figures}{reduction_kclique_n_v2}

\noindent Idea konstrukcji: \\

Chcemy aby krawędzie $p_.$ prowadziły do stanów $v_.$. Jeżeli krawędź $p_x$ prowadzi do stanu $v_i$, interpretujemy to jako wybranie wierzchołka $V_i$ jako $x$-ty wierzchołek w klice. \\

\begin{equation}
\forall x\in\{1..k\}\ \exists i\in\{1..n\} \ \delta(q_{\lambda},p_x)=v_i
\label{eq:kclique-vi}
\end{equation}

Do tej samej kliki nie możemy wybrać tego samego wierzchołka wielokrotnie, więc musimy zagwarantować, że dla różnych przejść $p_x$ i $p_y$ wybieramy różne stany $v_i$ i $v_j$. \\

\begin{equation}
\forall (x, y \in \{1..k\} \land x \neq y) \ \forall i, j \in \{1..n\} \ (\delta(q_{\lambda}, p_x) = v_i \land \delta(q_{\lambda}, p_y) = v_j) \implies i \neq j
\label{eq:kclique-diff}
\end{equation}

Dodatkowo musimy zagwarantować, że wybrane wierzchołki tworzą klikę, czyli że każdy wybrany wierzchołek jest połączony z każdym innym wybranym wierzchołkiem. \\

\begin{equation}
\forall (x, y \in \{1..k\} \land x \neq y) \ \forall{i, j \in \{1..n\}} \ (\delta(q_{\lambda}, p_x) = v_i \land \delta(q_{\lambda}, p_y) = v_j) \implies (v_i, v_j) \in E
\label{eq:kclique-clique}
\end{equation}


%* zagwarantowanie że wybieramy stany v_i
\noindent \textbf{Gwarancja prowadzenia brakujących krawędzi $p_x$ do wierzchołków $v_i$}

Zagwarantowanie tego faktu opiera się na prostej obserwacji: jedyne krawędzie $t$, które nie prowadzą do stanu odrzucającego, znajdują się za wierzchołkami $v_i$. Oznacza to, że aby próbka mogła zostać zaakceptowana, musimy przejść przez krawędź $p_x$ do któregoś ze stanów $v_i$, a następnie przez krawędź $t$ do stanu akceptującego. W przeciwnym wypadku, w trakcie przechodzenia próbką po automacie, trafimy do stanu $\text{sink}_-$, więc próbka zostanie odrzucona.\\
W takim przypadku, aby zagwarantować przejście krawędzią $p_x$ do któregoś ze stanów $v_i$, dodajemy w każdej próbce literę $t$ zaraz po literze $p_x$.

Jedynym wyjątkiem jest stan $\text{sink}_+$. 
Wychodząca z niego krawędź $t$ musi też prowadzić do $\text{sink}_+$ - w przeciwnym wypadku próbki \eqref{eq:kclique-n-clique} zostaną odrzucone gdy $p_x, p_y, v_i, v_j$ takich, że $(i, j) \notin E \implies \exists z \in \{1..n\} v_z = \delta(q_{\lambda}, p_x) \land z \neq i$, ponieważ ze stanu $v_z$ przejdziemy wtedy w automacie krawędzią $t$ a następnie $e_i$, a $z \neq i \implies \delta(\delta(v_z, t), e_i) = \text{sink}_+$. Ta próbka nie oznacza niepoprawnego przypisana krawędzi (więc powinna zostać zaakceptowana), a wiemy że w naszej próbce występuje jeszcze jedno przejście krawędzią $t$, więc musi ono prowadzić do stanu $\text{sink}_+$. \\
Możemy ten problem rozwiązać na 2 sposoby. Korzystając z faktu, że w próbce \eqref{eq:kclique-n-clique} są dokładnie 2 wystąpienia krawędzi $t$, możemy stworzyć alternatywny $\text{sink}_+$ (składający się z 2 stanów), który po pierwszym przejściu krawędzią $t$ prowadzi do drugiego stanu $\text{sink}_+$, a po drugim przejściu krawędzią $t$ prowadzi do stanu odrzucajcego $\text{sink}_-$. \\
Alternatywną metodą jest odrzucenie prowadzenie krawędzi $p_x$ do stanu $\text{sink}_+$ poprzez dodanie próbek odrzucających, które wymuszają prowadzenie krawędzi $p_x$ do stanu odrzucajcego:

\begin{equation}
\forall x\in\{1..k\} \ {[p_x] \in S^-}
\label{eq:kclique-n-sink}
\end{equation}

%* zagwarantowanie że wybrane stany v_i są różne
\noindent \textbf{Wybór różnych stanów $v_i$ dla różnych przejść $p_x$}

Aby uniemożliwić wybór tego samego $v_i$ dla różnych przejść $p_x$ oraz $p_y$ wystarczy stworzyć próbki postaci: 

\begin{equation}
\forall (x, y \in \{1..k\} \land x \neq y) \ \forall i \in \{1..n\} \ {[p_x \ t \ e_i \ c_1 \ p_y \ t \ e_i \ c_1]  \in S^+}
\label{eq:kclique-n-diff}
\end{equation}

W ten sposób, jeżeli oba przejścia $p_x$ i $p_y$ prowadziłyby do tego samego stanu $v_i$, to po pierwszym przejściu krawędziami $t \ e_i c_1$ trafilibyśmy do stanu $q_{\lambda}$, a następnie po drugim przejściu krawędziami $t \ e_i$ trafilibyśmy do stanu odrzucajcego $q_{\lambda}$, więc próbka zostałaby odrzucona. \\
Jeżeli jednak przejścia $p_x$ i $p_y$ prowadzą do różnych stanów $v_i$ i $v_j$, to pierwsze lub drugie przejście krawędziami $t \ e_i$ doprowadzi do stanu $q_{\lambda}$, a następnie przejście krawędzią $c_1$ doprowadzi do stanu akceptującego $\text{sink}_+$, więc próbka zostanie zaakceptowana.

\vspace{0.3cm}

%* zagwarantowanie że wybrane wierzchołki tworzą klikę
\noindent \textbf{Gwarancja, że wybrane wierzchołki tworzą klikę}

Aby zagwarantować, że wybrane wierzchołki tworzą klikę, tworzymy próbki wymuszające sąsiedztwo wybranych wierzchołków:

\begin{equation}
\forall (x, y \in \{1..k\} \land x \neq y) \ \forall{i, j \in \{1..n\}} \ [p_x \ t \ e_i \ c_1 \ p_y \ t \ e_j \ c_2 \ e_i] \in S^+
\label{eq:kclique-n-clique}
\end{equation}


Jeżeli dany stan nie sprawdza wyboru wierzchołków $V_i$ lub $V_j$ w grafie, to któraś z wybranych krawędzi $p_x$ lub $p_y$ nie prowadzi do stanu $v_i$ lub $v_j$. Wtedy przejście krawędziami $t \ e_i$ lub $t \ e_j$ doprowadzi do stanu akceptującego $\text{sink}_+$. \\
Jeżeli mamy jednak sytuację, w której $\delta(q_{\lambda}, p_x) = v_i$ oraz $\delta(q_{\lambda}, p_y) = v_j$, to przejście krawędziami $t \ e_i$ doprowadzi do stanu $u_i$, skąd krawędzią $c_1$ trafimy do $q_{\lambda}$. 
Przejście krawędziami $t \ e_j$ doprowadzi do stanu $u_j$. Ze stanu $u_j$ przejście krawędziami $c_2 \ e_i$ doprowadzi do stanu akceptującego tylko wtedy, gdy w oryginalnym grafie istnieje krawędź $(v_j, v_i) \in E$. W przeciwnym wypadku przejście krawędziami $c_2 \ e_i$ doprowadzi do stanu odrzucającego. \\


\noindent \textbf{Złożoność konstrukcji}

Rozmiar automatu jest wielomianowy w stosunku do rozmiaru grafu wejściowego - posiada on $O(n^2)$ stanów oraz $n$-krotnie więcej krawędzi, z racji iż $|\Sigma|=O(n)$. \\
Czas konstrukcji automatu jest również wielomianowy w stosunku do rozmiaru konstruowanego automatu.

Liczba próbek jest wielomianowa i wynosi $O(k^2 n^2)$, a ich łączna długość jest stała $O(1)$.


\subsubsection{Redukcja dla alfabetu o rozmiarze 3}

Aby zredukować liczbę potrzebnych liter z $|\Sigma|=O(n)$ do $|\Sigma|=3$, możemy zakodować każdą literę z oryginalnego alfabetu jako unikalny ciąg liter z alfabetu 2-literowego (rysunek \ref{fig:kclique3}). W tym celu użejemy kodowania binarnego, gdzie każda litera z oryginalnego alfabetu jest reprezentowana jako ciąg liter '0' i '1' o długości $\lceil \log_2(n) \rceil$. \\
Ustalimy więc konwencję zapisu, w której dla danego $a$ krawędź z etykietą $bin(a)$ reprezentuje ciąg stanów połączonych krawędziami o etykietach odpowiadających kolejnym bitom w kodzie binarnym $bin(a)$. Jeżeli mamy wiele krawędzi $bin$ wychodzących z tego samego wierzchołka, to ich krawędzie o tych samych etykietach stanowią tą samą krawędź. Na przykład, jeżeli mamy litery $a$ oraz $b$ reprezentowane przez $101$ oraz $100$, reprezentują one 4 krawędzie - najpierw ciąg dwóch, gdzie pierwsza ma etykietę $1$, a druga $0$ - następnie krawędzie $1$ oraz $0$ wychodzące z ostatniego wierzchołka (tak jak na rysunku \ref{fig:kclique-bin-example}).

\import{theory/figures}{reduction_kclique_bin_example}

\import{theory/figures}{reduction_kclique_3}

Idea działania oraz konstrukcja próbek pozostaje taka sama jak w przypadku alfabetu o rozmiarze $O(n)$, z tą różnicą, że każda krawędź jest teraz reprezentowana jako ciąg krawędzi zgodnie z powyższą konwencją. \\
Dodatkowo nie będziemy już korzystać z litery $t$, a litery $c_0$ i $c_1$ zastąpimy literą $c$, po której nastąpi litera $0$ lub $1$. \\

W poniższym zapisie $bin(a)$ oznacza ciąg liter reprezentujący literę $a$ w kodzie binarnym.

\vspace{0.3cm}

%* zagwarantowanie że wybieramy stany v_i
\noindent \textbf{Gwarancja prowadzenia brakujących krawędzi do wierzchołków $v_i$}

Zagwarantowanie tego faktu opiera się na tej samej obserwacji co w przypadku alfabetu o rozmiarze $O(n)$. Aby próbka mogła zostać zaakceptowana, musimy dwukrotnie przejść przez krawędź $c$ - dodamy więc w każdej próbce literę $c$ po każdym wyborze wierzchołka $v_i$. 

Teraz jedyne stany do których może prowadzić wybrakowana krawędź to stany $v_i$, stan $u_i$, stan $q_\lambda$ oraz stan $\text{sink}_+$. \\
Jeżeli brakująca krawędź prowadzi do $u_i$, to następna litera $1$ lub $0$ doprowadzi do stanu odrzucającego. \\
Aby zapobiec prowadzeniu brakujących krawędzi do stanu $sink_+$, możemy skorzystać z tej samej metody co poprzednio - stworzyć odpowiednie odrzucające próbki:

\begin{equation}
\forall x\in\{1..k\} \ {[bin(x) \ 0] \in S^-}
\label{eq:kclique-3-sink}
\end{equation}

Jeżeli brakująca krawędź prowadzić do stanu $q_\lambda$, to następny ciąg liter $bin(i)$ doprowadzi do stanu odrzucającego jeżeli $k \le i$. W przeciwnym wypadku doprowadzi nas do stanu $v_i$, skąd następna litera $c$ doprowadzi do stanu odrzucającego. \\

%* zagwarantowanie że wybrane stany v_i są różne
\noindent \textbf{Wybór różnych stanów $v_i$ dla różnych brakujących krawędzi}
\begin{equation}
\forall (x, y \in \{1..k\} \land x \neq y) \ \forall i \in \{1..n\} \ {[bin(x) \ 0 \ t \ bin(i) \ 0 \ bin(y) \ 0 \ t \ bin(i) \ 0]  \in S^+}
\label{eq:kclique-3-diff}
\end{equation}

%* zagwarantowanie że wybrane wierzchołki tworzą klikę
\noindent \textbf{Gwarancja, że wybrane wierzchołki tworzą klikę}
\begin{equation}
\forall (x, y \in \{1..k\} \land x \neq y) \ \forall{i, j \in \{1..n\}} \ [bin(x) \ 0 \ t \ bin(i) \ 0 \ bin(y) \ 0 \ t \ bin(j) \ 1 \ bin(i)] \in S^+
\label{eq:kclique-3-clique}
\end{equation}


\noindent \textbf{Złożoność konstrukcji}

Rozmiar automatu jest wielomianowy w stosunku do rozmiaru grafu wejściowego - posiada on $O(n^2 * \log n)$ stanów oraz $3$-krotnie więcej krawędzi. \\
Czas konstrukcji automatu jest również wielomianowy w stosunku do rozmiaru konstruowanego automatu.

Liczba próbek jest wielomianowa i wynosi $O(k^2 n^2)$, a ich łączna długość wynosi $O(\log n)$.


\subsubsection{Redukcja dla alfabetu o rozmiarze 2}
Rysunek \ref{fig:kclique2} przedstawia konstrukcję automatu dla redukcji z k-klik, korzystająca z alfabetu o rozmiarze 2. \\
Konstrukcja próbek dla zagwarantowania wyboru różnych stanów $v_i$ oraz zagwarantowania, że wybrane wierzchołki tworzą klikę pozostaje analogiczna do poprzednich. Nie posiadamy już litery testowej $t$, więc musimy zmodyfikować zagwarantowanie prowadzenia brakujących krawędzi do wierzchołków $v_i$. \\

\import{theory/figures}{reduction_kclique_2}

Nowa idea automatu (rysunek \ref{fig:kclique2}) opiera się na przypisywaniu literom wartości od 1 do $n$, a następnie kodowaniu ich w systemie unarnym. W ten sposób każda litera jest reprezentowana jako ciąg liter $0$. \\
W poniższym zapisie $un(a)$ oznacza reprezentację litery $a$ w kodzie unarnym. Aby rozróżniać przejścia literą $1$, będziemy je składać z różnych wielokrotności $1$. Dlategu ustalamy zapis $d * a$ jako powtórzenie litery $a$ dokładnie $d$ razy.


\vspace{0.3cm}

%* zagwarantowanie że wybieramy stany v_i
\noindent \textbf{Gwarancja prowadzenia brakujących krawędzi do wierzchołków $v_i$}

Zaczniemy od zagwarantowania, że brakujące krawędzie nie prowadzą do żadnego stanu akceptującego, w tym do stanu $\text{sink}_+$. \\

\begin{equation}
\forall x\in\{1..k\} \ {[un(x) \ 1] \in S^-}
\label{eq:kclique-2-sink}
\end{equation}

Nie mamy już litery testowej $t$, więc musimy zmodyfikować automat oraz próbki tak, aby wymusić przejście brakującą krawędzią do któregoś ze stanów $v_i$. Możemy zastąpić literę $t$ specjalnym ciągiem $10*1$, który nie prowadzi do $\text{sink}_-$. Wtedy ten sam ciąg $10*1$ wystąpi dwukrotnie w każdej próbce, zaraz po przejściu brakującą krawędzią. 

Weźmy dowolną próbkę \eqref{eq:kclique-2-clique}, oraz odpowiadające jej $x, y, i, j$. Jeżeli brakująca krawędź odpowiadająca stanowi $p_y$ nie prowadzi do żadnego ze stanów $v_.$, to nie istnieje żaden inny stan do którego mogłaby ona prowadzić - musi prowadzić do stanu, po którym następuje przejście $11*1$, jedyne takie przejścia to $v_.$ oraz $\text{sink}_+$. Jednak przejście do $\text{sink}_+$ zostało wykluczone próbką \eqref{eq:kclique-2-sink}, więc pozostaje tylko przejście do któregoś ze stanów $v_.$.

Jeżeli brakująca krawędzi odpowiadająca stanowi $p_x$ nie prowadzi do żadnego ze stanów $v_.$, to może prowadzić tylko do stanów po których następuje przejście $10*1$. Jedynymi takimi stanami są stany $v_.$, $\delta(v_., 1)$ oraz $\text{sink}_+$ (który został wykluczony próbką \eqref{eq:kclique-2-sink}). Przejście do stanu $\delta(v_., 1)$, a następnie kolejnymi literami z próbik - $un(i)$ doprowadzi nas do krawędzi $3*1$ lub $4*1$, natomiast w próbce występuje $2*1$, a po nim $un(y)$ - czyli litera $0$. To przejście zostanie więc zawszez odrzucone. Pozostaje więc tylko przejście do któregoś ze stanów $v_.$. 

Pozostaje jeszcze przypadek, w którym brakująca krawędź prowadzi do stanu startowego $q_\lambda$. Aby odrzucić taki przypadek, dodajemy próbki odrzucające:

\begin{equation}
\forall x\in\{1..k\} \ {[un(x) \ 1] \in S^-}
\label{eq:kclique-2-start}
\end{equation}

%* zagwarantowanie że wybrane stany v_i są różne
\noindent \textbf{Wybór różnych stanów $v_i$ dla różnych brakujących krawędzi}
\begin{equation}
\forall (x, y \in \{1..k\} \land x \neq y) \ \forall i \in \{1..n\} \ {[un(x) \ 11*1 \ un(i) \ 2*1 \ un(y) \ 11*1 \ un(i) \ 2*1]  \in S^+}
\label{eq:kclique-2-diff}
\end{equation}

%* zagwarantowanie że wybrane wierzchołki tworzą klikę
\noindent \textbf{Gwarancja, że wybrane wierzchołki tworzą klikę}
\begin{equation}
\forall (x, y \in \{1..k\} \land x \neq y) \ \forall{i, j \in \{1..n\}} \ [un(x) \ 11*1 \ un(i) \ 2*1 \ un(y) \ 12*1 \ un(j) \ 4*1 \ un(i) \ 1] \in S^+
\label{eq:kclique-2-clique}
\end{equation}


\noindent \textbf{Złożoność konstrukcji}

Rozmiar automatu jest wielomianowy w stosunku do rozmiaru grafu wejściowego - posiada on $O(n^2)$ stanów oraz $2$-krotnie więcej krawędzi. \\
Czas konstrukcji automatu jest również wielomianowy w stosunku do rozmiaru konstruowanego automatu.

Liczba próbek jest wielomianowa i wynosi $O(k^2 n^2)$, a ich łączna długość wynosi $O(n)$.
