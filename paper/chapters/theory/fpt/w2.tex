W tym rozdziale pokażemy, że problem naprawienia częściowego automatu jest $W[2]$-trudny. Do tego celu zredukujemy następujący problem $W[2]$ zupełny.

\begin{definition}[Problem k-zbioru pokrywającego (k-set cover)]
Weźmy rodzinę zbiorów $\mathcal{F}$ stworzoną na uniwersum $U$. Dla podrodziny $\mathcal{F}' \subset \mathcal{F}$ oraz podzbioru $U' \subset U$ mówimy, że $\mathcal{F}'$ pokrywa $U'$, jeżeli $\bigcup_{S \in \mathcal{F}'} S \supseteq U'$. W problemie k-zbioru pokrywającego dane jest uniwersum $U$, rodzina zbiorów $\mathcal{F}$ stworzona na $U$ oraz liczba naturalna $k$. Celem jest znalezienie podrodziny $\mathcal{F}' \subseteq \mathcal{F}$, takiej że $|\mathcal{F}'| \leq k$ oraz $\mathcal{F}'$ pokrywa $U$.

\noindent Problem k-zbioru pokrywającego należy do klasy $W[2]$-zupełnych problemów \cite{parametrized_algorithms}.
\end{definition}


\begin{theorem}
k-zbiór pokrywający $\leq_{\mathrm{FPT}}$ naprawienie k-częściowego DFA
\end{theorem}

\begin{proof}
Pokażemy redukcję z problemu k-zbioru pokrywającego do problemu naprawienia częściowego DFA. Dla dowolnego wejścia do problemu k-zbioru pokrywającego skonstruujemy automat oraz próbki, będące wejściem do problemu naprawienia częściowego DFA, tak aby rozwiązanie problemu naprawienia częściowego DFA rozwiązywało problem k-zbioru pokrywającego. 

Automat \ref{fig:ksetcover} z redukcji posiada rozmiar alfabetu zależny od rozmiaru uniwersum $U$ - można go w analogiczny sposób jak w przypadku redukcji z k-klik zmniejszyć do stałego rozmiaru alfabetu. \\
Przyjmijmy też, że $n = |\mathcal{F}|$. \\

\noindent Idea konstrukcji: \\
Każdy stan $s_i$ odpowiada zbiorowi $S_i \in \mathcal{F}$. Krawędzie $e_j$ odpowiadają elementom $u_j \in U$. \\
Jeżeli brakująca krawędź $p_x$ prowadzi do stanu $s_i$, to interpretujemy to jako wybranie zbioru $S_i$ jako $x$-tego zbioru w pokryciu. \\
Sekcja znajdująca się za wierzchołkami $s_i$ prowadzi do stanu startowego (odrzucającego) $q_{\lambda}$, jeżeli dany zbiór $S_i$ nie zawiera elementu $u_j$. W przeciwnym wypadku prowadzi do stanu akceptującego $\text{sink}_+$. Konstrukcja ta reprezentuje elementu z uniwersum $U$, które znajdują się w zbiorze $S_i$.

\import{theory/figures/w2}{reduction_ksetcover_n}

W tym problemie nie musimy gwarantować wyboru różnych stanów $s_i$ dla różnych krawędzi $p_x$, ponieważ wybór tego samego zbioru wielokrotnie nie ma wpływu na pokrycie uniwersum.

\noindent \textbf{Gwarancja pokrycia uniwersum}

Aby zagwarantować, że wybrane zbiory pokrywają całe uniwersum, dla każdego elementu $u_j \in U$ tworzymy próbkę przechodzącą do każdego, wybranego przez krawędź $p_x$, stanu $s_i$ oraz sprawdzającą czy dany element $u_j$ znajduje się w zbiorze $S_i$. Jeżeli tak, próbka jest akceptowana, w przeciwnym wypadku odrzucana. \\

\begin{equation}
\forall j \in \{1..|U|\} \ {[p_1 \ t \ e_j \ p_2 \ t \ e_j \ ... \ p_k \ t \ e_j] \in S^+}
\label{eq:kcover-cover}
\end{equation}

Dodatkowo, w analogiczny sposób jak w przypadku redukcji z k-klik, tworzymy próbki wymuszające prowadzenie brakujących krawędzi do stanów $s_i$ oraz wybór różnych stanów dla różnych krawędzi $p_x$.


\noindent \textbf{Gwarancja prowadzenia brakujących krawędzi do stanów $s_i$}

Poprowadzenie brakującej krawędzi do dowolnego stanu który nie jest $s_i$ lub $\text{sink}_+$ doprowadzi do $\text{sink}_-$ po przejściu następną literą $t$.\\
Aby zapobiec prowadzeniu brakujących krawędzi do stanu $\text{sink}_+$, dodajemy próbki odrzucające:

\begin{equation}
\forall x\in\{1..k\} \ {[p_x] \in S^-}
\label{eq:kcover-si}
\end{equation}

\end{proof}


\noindent \textbf{Złożoność konstrukcji}

Rozmiar automatu jest wielomianowy w stosunku do rozmiaru danych wejściowych - posiada on $O(n^2)$ stanów oraz $n$-krotnie więcej krawędzi. \\
Czas konstrukcji automatu jest również wielomianowy w stosunku do rozmiaru konstruowanego automatu.

Liczba próbek jest wielomianowa i wynosi $O(k + |U|)$, a ich łączna długość wynosi $O(k)$.




Powyższy dowód wymaga użycia alfabetu o arności liniowej względem $k$. Poniżej pokażemy jednak, że rozmiar alfabetu nie ma znaczenia, dostosowując redukcję do alfabetu rozmiaru $2$.

Automat nazwiemy binarnym, jeśli jest nad alfabetem ${0, 1}$.



\begin{theorem}
k-zbiór pokrywający $\leq_{\mathrm{FPT}}$ naprawianie k-częściowego automatu binarnego
\end{theorem}

\begin{proof}

W konstrukcji z resunku \ref{fig:ksetcover2} zamieniamy krawędzie $p_x$ na unarne ciągi stanów, gdzie każdy stan jest połączony przejściem etykietowanym $0$ do następnego stanu w ciągu. Dla i-tego stanu w ciągu, krawędź etykietowana $1$ jest wybrakowaną krawędzią prowadzącą (którą będziemy chcieli prowadzić tylko i wyłącznie do stanów $s_i$). Długość ciągu wynosi $k$, bo i-ty stan w ciągu odpowiada krawędzi $p_i$ z poprzedniej konstrukcji, a takich krawędzi jest $k$.

Następnie z każdego stanu $s_i$ wychodzi ciąg $10$ krawędzi $1$, odpowiadających krawędzi testowej (taki ciąg nie wystąpuje nigdzie indziej w automacie, poza stanem $\text{sink}_+$). 

Po krawędzi testowej ponownie mamy unarne odliczanie $n$ możliwych stanów, odpowiadajcym krawędziom $e_j$ z poprzedniej konstrukcji. Z każdego stanu w tym ciągu wychodzi krawędź etykietowana $1$, prowadząca do stanu odrzucającego $q_{\lambda}$ lub akceptującego $\text{sink}_+$ - zależnie od tego, czy dany element $u_j$ należy do zbioru $S_i$.


\import{theory/figures/w2}{reduction_ksetcover_2}


\noindent \textbf{Gwarancja pokrycia uniwersum}

Aby zagwarantować, że wybrane zbiory pokrywają całe uniwersum, dla każdego elementu $u_j \in U$ tworzymy próbkę przechodzącą przez unarne odliczanie do każdego, wybranego przez krawędź w ciągu odpowiadającym $p_x$, stanu $s_i$ oraz sprawdzającą czy dany element $u_j$ znajduje się w zbiorze $S_i$. Jeżeli tak, próbka jest akceptowana, w przeciwnym wypadku odrzucana. \\
Powtórzenie danej litery $a$ $c$-krotnie będziemy zapisywać jako $a^{c}$.

\begin{equation}
\forall j \in \{1..|U|\} \ {[0^{1} \ 1^{11} \ 0^{j} \ 1 \ 0^{2} \ 1^{11} \ 0^{j} \ 1 \ \dots \ 0^{k} \ 1^{11} \ 0^{j} \ 1] \in S^+}
\label{eq:kcover-cover-binary}
\end{equation}


\noindent \textbf{Gwarancja prowadzenia brakujących krawędzi do stanów $s_i$}

Poprowadzenie brakującej krawędzi do dowolnego stanu który nie jest $s_i$ lub $\text{sink}_+$ doprowadzi do $\text{sink}_-$ po przejściu następującym ciągiem liter $1^{10}$.\\
Aby zapobiec prowadzeniu brakujących krawędzi do stanu $\text{sink}_+$, dodajemy próbki odrzucające:

\begin{equation}
\forall x\in\{1..k\} \ {[0^{x}1] \in S^-}
\label{eq:kcover-si-binary}
\end{equation}


\end{proof}


\noindent \textbf{Złożoność konstrukcji}

Rozmiar automatu jest wielomianowy w stosunku do rozmiaru danych wejściowych - posiada on $O(n^2)$ stanów oraz $2$-krotnie więcej krawędzi. \\
Czas konstrukcji automatu jest również wielomianowy w stosunku do rozmiaru konstruowanego automatu.




