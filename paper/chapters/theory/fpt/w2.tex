W tym rozdziale pokażemy, że problem naprawienia $k$-częściowego automatu jest $W[2]$-trudny. Do tego celu zredukujemy następujący problem $W[2]$ zupełny.

\begin{definition}[Problem $k$-zbioru pokrywającego ($k$-set cover)]
Weźmy rodzinę zbiorów $\mathcal{F}$ stworzoną na uniwersum $U$. Dla podrodziny $\mathcal{F}' \subset \mathcal{F}$ oraz podzbioru $U' \subset U$ mówimy, że $\mathcal{F}'$ pokrywa $U'$, jeżeli $\bigcup_{S \in \mathcal{F}'} S \supseteq U'$. W~problemie $k$-zbioru pokrywającego dane jest uniwersum $U$, rodzina zbiorów $\mathcal{F}$ stworzona na $U$ oraz liczba naturalna $k$. Celem jest znalezienie podrodziny $\mathcal{F}' \subseteq \mathcal{F}$, takiej że $|\mathcal{F}'| \leq k$ oraz $\mathcal{F}'$ pokrywa $U$.

\noindent Problem $k$-zbioru pokrywającego należy do klasy $W[2]$-zupełnych problemów \cite{parametrized_algorithms}.
\end{definition}


\begin{theorem}
$k$-zbiór pokrywający $\leq_{\mathrm{FPT}}$ naprawienie $k$-częściowego DFA
\end{theorem}

\begin{proof}
Pokażemy redukcję z~problemu $k$-zbioru pokrywającego do problemu naprawienia $k$-częściowego DFA. Dla dowolnego wejścia do problemu $k$-zbioru pokrywającego skonstruujemy automat oraz próbki, będące wejściem do problemu naprawienia $k$-częściowego DFA, tak aby rozwiązanie problemu naprawienia $k$-częściowego DFA rozwiązywało problem $k$-zbioru pokrywającego. 

Idea konstrukcji jest przedstawiona na rysunku \ref{fig:ksetcover}. Na tym rysunku $n$ oznacza moc rodziny $\mathcal{F}$, a $k$ parametr problemu. \\
Automat z~rysunku \ref{fig:ksetcover} z~redukcji posiada rozmiar alfabetu zależny od rozmiaru uniwersum $U$ - w dalszej części pracy udowodnimy, że rozmiar alfabetu można zmniejszyć do $2$.

W poniższych automatach stan startowy oznaczony jest jako $q_{\lambda}$, stany akceptujące zaznaczone są podwójną linią, a stan odrzucający pojedynczą. \\
Wybrakowane krawędzie zaznaczone są przerywaną linią (i opisane jako $p_x$), natomiast stany $q_\lambda$ zaznaczone przerywaną linią reprezentują ten sam stan startowy automatu. \\
Część automatu zaznaczona w ramce jest powtórzona dla każdego elementu $u_j \in U$. \\
Każda krawędź, która nie jest zaznaczona na rysunku, prowadzi do stanu odrzucającego $\text{sink}_-$. \\

\noindent \textbf{Idea konstrukcji:} \\
Każdy stan $s_i$, w automacie z~rysunku \ref{fig:ksetcover}, odpowiada zbiorowi $S_i \in \mathcal{F}$. Krawędzie $e_j$ odpowiadają elementom $u_j \in U$. \\
Jeżeli brakująca krawędź $p_x$ prowadzi do stanu $s_i$, to interpretujemy to jako wybranie zbioru $S_i$ jako $x$-tego zbioru w pokryciu. \\
Sekcja znajdująca się za wierzchołkami $s_i$ prowadzi do stanu startowego (odrzucającego) $q_{\lambda}$, jeżeli dany zbiór $S_i$ nie zawiera elementu $u_j$. W~przeciwnym wypadku prowadzi do stanu akceptującego $\text{sink}_+$. Konstrukcja ta reprezentuje elementu z~uniwersum $U$, które znajdują się w zbiorze $S_i$.

\import{theory/figures/w2}{reduction_ksetcover_n}

Możemy zaobserwować, że w tym problemie nie musimy gwarantować wyboru różnych stanów $s_i$ dla różnych krawędzi $p_x$, ponieważ wybór tego samego zbioru wielokrotnie nie ma wpływu na pokrycie uniwersum. Redukcja nie zmienia wartości $k$.


\vspace{0.3cm}

\noindent \textbf{Gwarancja pokrycia uniwersum}

Aby zagwarantować, że wybrane zbiory pokrywają całe uniwersum, dla każdego elementu $u_j \in U$ tworzymy próbkę przechodzącą do każdego, wybranego przez krawędź $p_x$, stanu $s_i$ oraz sprawdzającą czy dany element $u_j$ znajduje się w zbiorze $S_i$. Jeżeli tak, próbka jest akceptowana, w przeciwnym wypadku odrzucana. \\

\begin{equation}
\text{dla każdego } j \in \{1..|U|\} \ {[p_1 \ t \ e_j \ p_2 \ t \ e_j \ ... \ p_k \ t \ e_j] \in S^+}
\label{eq:kcover-cover}
\end{equation}

Dodatkowo, tworzymy próbki wymuszające prowadzenie brakujących krawędzi do stanów $s_i$.


\vspace{0.3cm}

\noindent \textbf{Gwarancja prowadzenia brakujących krawędzi do stanów $s_i$}

Poprowadzenie brakującej krawędzi do dowolnego stanu który nie jest $s_i$ lub $\text{sink}_+$ doprowadzi do $\text{sink}_-$ po przejściu następną literą $t$.\\
Aby zapobiec prowadzeniu brakujących krawędzi do stanu $\text{sink}_+$, dodajemy próbki odrzucające:

\begin{equation}
\text{dla każdego } x\in\{1..k\} \ {[p_x] \in S^-}
\label{eq:kcover-si}
\end{equation}


\vspace{0.5cm}


Teraz udowodnimy, że tak skonstruowana instancja problemu naprawienia $k$-częściowego DFA ma rozwiązanie wtedy i tylko wtedy, gdy oryginalna instancja problemu $k$-zbioru pokrywającego ma rozwiązanie.

\noindent \textbf{Istnieje rozwiązanie $k$-zbioru pokrywającego $\implies$ istnieje rozwiązanie naprawienia $k$-częściowego DFA}

Załóżmy, że istnieje rozwiązanie problemu $k$-zbioru pokrywającego, czyli istnieje podrodzina $\mathcal{F}' \subseteq \mathcal{F}$, taka że $|\mathcal{F}'| \leq k$ oraz $\mathcal{F}'$ pokrywa $U$. Wybierzmy dowolne uporządkowanie zbiorów w $\mathcal{F}'$ jako $S_{i_1}, S_{i_2}, \dots, S_{i_m}$, gdzie $m \leq k$.

Utworzymy teraz przypisanie brakujących krawędzi w automacie z~rysunku \ref{fig:ksetcover}, tak aby powstał poprawny pełny automat deterministyczny oraz aby $\text{każde } s \in S^+$ było akceptowane, a $\text{każde } s \in S^-$ odrzucane.

Dla każdej brakującej krawędzi $p_x$, gdzie $x \in \{1..m\}$, poprowadźmy ją do stanu $s_{i_x}$, odpowiadającemu zbiorowi $S_{i_x}$ z~pokrycia. Dla pozostałych brakujących krawędzi $p_x$, gdzie $x \in \{m+1..k\}$, poprowadźmy je do dowolnego stanu $s_i$.

Próbki \ref{eq:kcover-cover} będą akceptowane, ponieważ dla każdego elementu $u_j \in U$ istnieje zbiór $S_{i_x} \in \mathcal{F}'$ taki że $u_j \in S_{i_x}$, a zatem po pewnej liczbie przejść podsłowami $\text{dla } y \in \{1..x-1\} \ [p_y \ t \ e_j]$ prowadzącymi do stanu $q_{\lambda}$, przejdziemy podsłowem $[p_x \ t \ e_j]$, gdzie $p_x$ prowadzi do stanu $s_{i_x}$, więc następujące krawędzie $[t \ e_j]$ doprowadzą do stanu akceptującego $\text{sink}_+$ (ponieważ, z~konstrukcji automatu, $u_j \in S_{i_x} \implies \delta(\delta(s_{i_x}, t), e_j) = \text{sink}_+$).

Próbki \ref{eq:kcover-si} będą odrzucane, ponieważ każda brakująca krawędź $p_x$ prowadzi do stanu $s_i$ dla pewnego $i \in \{1..n\}$, a z~konstrukcji automatu $s_i \in \mathbb{F_R}$.


\vspace{0.5cm}

\noindent \textbf{Istnieje rozwiązanie naprawienia $k$-częściowego DFA $\implies$ istnieje rozwiązanie $k$-zbioru pokrywającego}

Załóżmy, że istnieje rozwiązanie problemu naprawienia $k$-częściowego DFA, czyli istnieje przypisanie brakujących krawędzi w automacie z~rysunku \ref{fig:ksetcover}, takie że powstał poprawny pełny automat deterministyczny oraz aby $\text{każde } s \in S^+$ było akceptowane, a~$\text{każde } s \in S^-$ odrzucane.

Zauważmy, że z~próbek \ref{eq:kcover-si} wynika, że każda brakująca krawędź $p_x$ musi prowadzić do pewnego stanu odrzucającego, więc nie możemy prowadzić żadnej brakującej krawędzi do stanu $\text{sink}_+$. \\ 
Ponadto, z~próbek \ref{eq:kcover-cover} wynika, że krawędź $p_x$ nie może prowadzić do stanu $q_\lambda$, ani $\text{dla każdego } i \in \{1..n\}$ do stanu $\delta(s_i, t)$, ponieważ w obu przypadkach zawsze następna litera to $t$, która prowadzi do stanu odrzucającego $\text{sink}_-$. \\
Dodatkowo, prowadzenie krawędzi $p_x$ do stanu $\text{sink}_-$ prowadzi do odrzucenia próbek \ref{eq:kcover-cover}, ponieważ przejście każdą krawędzią $p_x$ występuje w każdej próbce - więc każda próbka zostanie odrzucona. \\
Zatem każda brakująca krawędź $p_x$ musi prowadzić do pewnego stanu $s_i$.

Stwórzmy teraz podrodzinę $\mathcal{F}' \subseteq \mathcal{F}$, wybierając zbiory $S_i$ odpowiadające stanom $s_i$, do których prowadzą brakujące krawędzie $p_x$. \\
Jeżeli istnieje wiele krawędzi $p_x$ prowadzących do tego samego stanu $s_i$, to wybieramy zbiór $S_i$ tylko raz. \\
Zauważmy, że z~powyższych obserwacji wynika, że $|\mathcal{F}'| \leq k$, ponieważ istnieje dokładnie $k$ brakujących krawędzi $p_x$.

Weźmy dowolny element $u_j \in U$. Odpowiadająca mu próbka \ref{eq:kcover-cover} musi być akceptowana przez automat. Oznacza to, że istnieje krawędź $p_x$ prowadząca do stanu $s_i$, takiego że $u_j \in S_i$ (ponieważ tylko w takim przypadku przejście następującą literą $t$ oraz $e_j$ doprowadzi do stanu akceptującego $\text{sink}_+$).

Zatem każdy element $u_j \in U$ jest zawarty w pewnym zbiorze $S_i \in \mathcal{F}'$, więc $\mathcal{F}'$ pokrywa $U$ - skonstruowane rozwiązanie problemu $k$-zbioru pokrywającego jest poprawne.


\vspace{0.5cm}

\noindent \textbf{Złożoność konstrukcji}

Rozmiar automatu jest wielomianowy w stosunku do rozmiaru danych wejściowych - posiada on $O(n^2)$ stanów oraz $n$-krotnie więcej krawędzi. \\
Czas konstrukcji automatu jest również wielomianowy w stosunku do rozmiaru konstruowanego automatu.

Liczba próbek jest wielomianowa i wynosi $O(k + |U|)$, a ich łączna długość wynosi $O(k \cdot |U|)$.


\vspace{0.5cm}

\noindent \textbf{Podsumowanie}

Powyższa konstrukcja jest redukcją parametryczną z~problemu $k$-zbioru pokrywającego do problemu naprawienia $k$-częściowego DFA, ponieważ rozmiar konstruowanego automatu oraz liczba próbek są wielomianowe względem rozmiaru wejścia, a liczba brakujących krawędzi w automacie wynosi dokładnie $k$.

Wykazaliśmy też, że istnienie rozwiązania jednego problemu jest równoważne istnieniu rozwiązania drugiego problemu.

Zatem pokazaliśmy, że $k$-zbiór pokrywający $\leq_{\mathrm{FPT}}$ naprawianie $k$-częściowego DFA.

\end{proof}


\vspace{0.5cm}


Powyższy dowód wymaga użycia alfabetu o arności liniowej względem $k$. Poniżej pokażemy jednak, że rozmiar alfabetu nie ma znaczenia, dostosowując redukcję do alfabetu rozmiaru $2$.

Automat nazwiemy binarnym, jeśli jest nad alfabetem $\{0, 1\}$.


\vspace{0.5cm}


\begin{theorem}
$k$-zbiór pokrywający $\leq_{\mathrm{FPT}}$ naprawianie $k$-częściowego automatu binarnego
\end{theorem}

\begin{proof}

W konstrukcji z~rysunku \ref{fig:ksetcover} dodajemy ciąg stanów, gdzie każdy stan jest połączony przejściem etykietowanym $0$ do następnego stanu w ciągu (rysunek \ref{fig:ksetcover2}). 
Dla $i$-tego stanu w ciągu, krawędź etykietowana $1$ jest wybrakowanym przejściem w automacie - które będziemy chcieli prowadzić tylko i wyłącznie do stanów $s_i$. 
Długość ciągu wynosi $k$, bo $i$-ty stan w ciągu odpowiada krawędzi $p_i$ z~poprzedniej konstrukcji, a takich krawędzi jest $k$.

Z każdego stanu $s_i$ zamiast krawędzi etykietowanej $t$ wychodzi krawędź etykietowana $1$, która prowadzi do ciągu $10$ stanów połączonych szeregowo krawędzią etykietowaną $1$ (krawędź $0$ wychodząca z~tych stanów prowadzi do $\text{sink}_-$). Taki ciąg nie występuje nigdzie indziej w automacie, poza stanem $\text{sink}_+$ - czyli jedyne stany w automacie, z~których można przejść słowem $[1^{10}]$ to $s_i$ oraz $\text{sink}_+$.

Po krawędzi testowej ponownie mamy unarne odliczanie $n$ możliwych stanów, odpowiadającym krawędziom $e_j$ z~poprzedniej konstrukcji. Z każdego stanu w tym ciągu wychodzi krawędź etykietowana $1$, prowadząca do stanu odrzucającego $q_{\lambda}$ lub akceptującego $\text{sink}_+$ - zależnie od tego, czy dany element $u_j$ należy do zbioru $S_i$.

Analogicznie jak w poprzedniej konstrukcji, każda krawędź nie pokazana na rysunku prowadzi do stanu odrzucającego $\text{sink}_-$. 

Powtórzenie danej litery $a$ $c$-krotnie będziemy zapisywać jako $a^{c}$.

\import{theory/figures/w2}{reduction_ksetcover_2}


\noindent \textbf{Gwarancja pokrycia uniwersum}

Aby zagwarantować, że wybrane zbiory pokrywają całe uniwersum, dla każdego elementu $u_j \in U$ tworzymy próbkę przechodzącą przez unarne odliczanie do każdego, wybranego przez krawędź w ciągu odpowiadającym $p_x$, stanu $s_i$ oraz sprawdzającą czy dany element $u_j$ znajduje się w zbiorze $S_i$. Jeżeli tak, próbka jest akceptowana, w~przeciwnym wypadku odrzucana. \\

\begin{equation}
\text{dla każdego } j \in \{1..|U|\} \ {[0^{1} \ 1^{11} \ 0^{j} \ 1 \ 0^{2} \ 1^{11} \ 0^{j} \ 1 \ \dots \ 0^{k} \ 1^{11} \ 0^{j} \ 1] \in S^+}
\label{eq:kcover-cover-binary}
\end{equation}


\noindent \textbf{Gwarancja prowadzenia brakujących krawędzi do stanów $s_i$}

Poprowadzenie brakującej krawędzi do dowolnego stanu który nie jest $s_i$ lub $\text{sink}_+$ doprowadzi do $\text{sink}_-$ po przejściu następującym ciągiem liter $1^{10}$.\\
Aby zapobiec prowadzeniu brakujących krawędzi do stanu $\text{sink}_+$, dodajemy próbki odrzucające:

\begin{equation}
\text{dla każdego } x\in\{1..k\} \ {[0^{x}1] \in S^-}
\label{eq:kcover-si-binary}
\end{equation}


\vspace{0.5cm}

\noindent \textbf{Istnieje rozwiązanie $k$-zbioru pokrywającego $\implies$ istnieje rozwiązanie naprawienia $k$-częściowego automatu binarnego}

Załóżmy, że istnieje rozwiązanie problemu $k$-zbioru pokrywającego, czyli istnieje podrodzina $\mathcal{F}' \subseteq \mathcal{F}$, taka że $|\mathcal{F}'| \leq k$ oraz $\mathcal{F}'$ pokrywa $U$. Wybierzmy dowolne uporządkowanie zbiorów w $\mathcal{F}'$ jako $S_{i_1}, S_{i_2}, \dots, S_{i_m}$, gdzie $m \leq k$.

Dla każdej brakującej krawędzi w ciągu odpowiadającym $p_x$, gdzie $x \in \{1..m\}$, poprowadźmy ją do stanu $s_{i_x}$, odpowiadającemu zbiorowi $S_{i_x}$ z~pokrycia. Dla pozostałych brakujących krawędzi w ciągu odpowiadającym $p_x$, gdzie $x \in \{m+1..k\}$, poprowadźmy je do dowolnego stanu $s_i$.

Próbki \ref{eq:kcover-cover-binary} będą akceptowane, ponieważ dla każdego elementu $u_j \in U$ istnieje zbiór $S_{i_x} \in \mathcal{F}'$ taki że $u_j \in S_{i_x}$, a zatem po pewnej liczbie przejść podsłowami  \\  $\text{dla każdego } y \in \{1..k\} \ [0^{y} \ 1^{11} \ 0^{j} \ 1]$  prowadzącymi do stanu $q_{\lambda}$, przejdziemy podsłowem $[0^{x} \ 1^{11} \ 0^{j} \ 1]$, gdzie $0^{x}$ prowadzi do stanu $s_{i_x}$, więc następujące krawędzie $[1^{11} \ 0^{j} \ 1]$ doprowadzą do stanu akceptującego $\text{sink}_+$ (ponieważ, z~konstrukcji automatu, $u_j \in S_{i_x} \implies \delta(\delta(s_{i_x}, 1^{11}), 0^{j}) = \text{sink}_+$).

Próbki \ref{eq:kcover-si-binary} będą odrzucane, ponieważ każda brakująca krawędź w~ciągu odpowiadającym $p_x$ prowadzi do stanu $s_i$ dla pewnego $i \in \{1..n\}$, a z~konstrukcji automatu $s_i \in \mathbb{F_R}$.


\vspace{0.5cm}

\noindent \textbf{Istnieje rozwiązanie naprawienia $k$-częściowego automatu binarnego $\implies$ istnieje rozwiązanie $k$-zbioru pokrywającego}

Załóżmy, że istnieje rozwiązanie problemu naprawienia $k$-częściowego automatu binarnego, czyli istnieje przypisanie brakujących krawędzi w~automacie z~rysunku \ref{fig:ksetcover2}, takie że powstał poprawny pełny automat deterministyczny oraz aby $\text{każde } s \in S^+$ było akceptowane, a $\text{każde } s \in S^-$ odrzucane.

Zauważmy, że z~próbek \ref{eq:kcover-si-binary} wynika, że każda brakująca krawędź wychodząca ze stanu $p_x$ musi prowadzić do pewnego stanu odrzucającego, więc nie możemy prowadzić żadnej brakującej krawędzi do stanu $\text{sink}_+$. \\
Ponadto, z~próbek \ref{eq:kcover-cover-binary} wynika, że brakująca krawędź nie może prowadzić do stanu $q_\lambda$, ponieważ następna litera w~próbce to $1$, która w~konstrukcji automatu prowadzi do stanu odrzucającego $\text{sink}_-$. \\
Dodatkowo, nie może ona prowadzić do żadnego stanu $\text{dla } i \in \{1..n\} \ e_i$, ponieważ następna litera w~próbce to $1$, która w~konstrukcji automatu prowadzi do stanu odrzucającego $\text{sink}_-$. \\
Jeżeli brakująca krawędź prowadzi do stanu $\text{dla } i \in \{1..n\} \ \text{dla } j \in \{1..10\} \ \delta(s_i, 1^{j})$, to w~próbce następuje po niej ciąg liter $[1^{10} \ 0]$, który będzie prowadzić do stanu odrzucającego $\text{sink}_-$. \\
Oznacza to też, że nie może ona prowadzić do żadnego stanu $\text{dla } i \in \{1..k\} \ p_i$, ponieważ zawsze następujący ciąg liter to $[ 1^{10} \ 0]$, który zawsze prowadzi do stanu odrzucającego $\text{sink}_-$.

Zatem każda brakująca krawędź w~ciągu odpowiadającym $p_x$ musi prowadzić do pewnego stanu $s_i$.

Stwórzmy teraz podrodzinę $\mathcal{F}' \subseteq \mathcal{F}$, wybierając zbiory $S_i$ odpowiadające stanom $s_i$, do których prowadzą brakujące krawędzie wychodzące ze stanów $p_x$. \\
Jeżeli istnieje wiele brakujących krawędzi prowadzących do tego samego stanu $s_i$, to wybieramy zbiór $S_i$ tylko raz. \\
Zauważmy, że z~powyższych obserwacji wynika, że $|\mathcal{F}'| \leq k$, ponieważ istnieje dokładnie $k$ brakujących krawędzi w~ciągach odpowiadających $p_x$.

Weźmy dowolny element $u_j \in U$. Odpowiadająca mu próbka \ref{eq:kcover-cover-binary} musi być akceptowana przez automat. Oznacza to, że istnieje brakująca krawędź w~ciągu odpowiadającym $p_x$ prowadząca do stanu $s_i$, takiego że $u_j \in S_i$ (ponieważ tylko w~takim przypadku przejście następującym ciągiem liter $1^{11} \ 0^{j} \ 1$ doprowadzi do stanu akceptującego $\text{sink}_+$).

Zatem każdy element $u_j \in U$ jest zawarty w~pewnym zbiorze $S_i \in \mathcal{F}'$, więc $\mathcal{F}'$ pokrywa $U$ - skonstruowane rozwiązanie problemu $k$-zbioru pokrywającego jest poprawne.

\vspace{0.5cm}

\noindent \textbf{Złożoność konstrukcji}

Rozmiar automatu jest wielomianowy w~stosunku do rozmiaru danych wejściowych - posiada on $O(n^2 + k)$ stanów oraz $2$-krotnie więcej krawędzi. \\
Czas konstrukcji automatu jest również wielomianowy w~stosunku do rozmiaru konstruowanego automatu.


\vspace{0.5cm}

\noindent \textbf{Podsumowanie}

Analogicznie jak w~poprzedniej konstrukcji, powyższa konstrukcja jest redukcją parametryczną z~problemu $k$-zbioru pokrywającego do problemu naprawienia $k$-częściowego automatu binarnego, ponieważ rozmiar konstruowanego automatu oraz liczba próbek są wielomianowe względem rozmiaru wejścia, liczba brakujących krawędzi w~automacie wynosi dokładnie $k$ oraz istnienie rozwiązania jednego problemu jest równoważne istnieniu rozwiązania drugiego problemu. 

Zatem pokazaliśmy, że $k$-zbiór pokrywający $\leq_{\mathrm{FPT}}$ naprawianie $k$-częściowego automatu binarnego.



\end{proof}





