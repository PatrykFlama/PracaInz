Dowiedziona $\mathbb{NP}$-zupełność problemu naprawienia częściowego automatu dotyczy przypadku, gdy z automatu usunięto wszystkie krawędzie. Jest to skrajna sytuacja. W ramach tej pracy chcieliśmy się skupić na sytuacji, w której liczba brakujących krawędzi jest ograniczona. Zauważmy, że jeżeli w automacie o n stanach brakuje tylko jednej krawędzi, to do jego naprawy wystarczy sprawdzić n automatów -- po jednym dla każdego możliwego końca tej krawędzi. Tę obserwację można uogólnić do następującego faktu.

\begin{theorem}
Dla każdego ustalonego k, problem naprawiania częściowego automatu ograniczony
do automatów z co najwyżej k brakującymi krawędziami jest w klasie problemów rozwiązywalnych w czasie wielomianowym (klasa $\mathbb{P}$).
\label{th:ptime}
\end{theorem}

\begin{proof}
Niech dany będzie częściowy automat deterministyczny $A$ o $n$ stanach z co najwyżej $k$ brakującymi krawędziami. Algorytm polega na wyliczeniu wszystkich możliwych uzupełnień automatu poprzez dodanie brakujących krawędzi.

Liczba wszystkich możliwych krawędzi w automacie to co najwyżej $n \cdot |\Sigma|$, gdzie $|\Sigma|$ to rozmiar alfabetu. Każda brakująca krawędź może być skierowana do dowolnego spośród $n$ stanów, zatem liczba możliwych uzupełnień to co najwyżej $n^k$.

Dla każdego z tych uzupełnień sprawdzamy w czasie wielomianowym (zależnym od rozmiaru automatu), czy powstały automat spełnia warunki specyfikacji. Ta weryfikacja wykonywana jest w czasie $O(n^c)$ dla pewnej stałej $c$.

Całkowita złożoność wynosi $O(n^k \cdot n^c) = O(n^{k+c})$, co jest wielomianem względem rozmiaru wejścia dla ustalonego $k$. Zatem problem należy do $\mathbb{P}$ dla każdego ustalonego $k$. 

\end{proof}


Złożoność parametryczna (ang. \textit{parameterized complexity}) to framework do analizy problemów obliczeniowych, w których oprócz rozmiaru wejścia istnieje dodatkowy parametr. 
Zamiast klasyfikacji problemów jedynie jako $\mathbb{P}$ lub $\mathbb{NP}$-zupełne, badamy zależność złożoności od tego parametru.

\vspace{0.5cm}

Z Twierdzenia \ref{th:ptime} można wywnioskować, że parametryczna wersja problemu naprawiania częściowego automatu należy do klasy XP, czyli klasy problemów, które można rozwiązać w czasie $n^{f(k)}$, gdzie $f(k)$ jest funkcją tylko od parametru $k$. 

Nas interesuje dogłębniejsze zbadanie parametrycznej złożoności tego problemu, a w szczególności ustalenie, czy problem ten należy do klasy FPT, która jest parametrycznym odpowiednikiem klasy $\mathbb{P}$. 

\begin{definition}[Klasa FPT]
Klasa FPT składa się z problemów, dla których istnieje algorytm rozwiązujący je w czasie $f(k) \cdot n^c$, gdzie $n$ to rozmiar wejścia, $f$ to dowolna funkcja obliczalna, a $c$ to stała niezależna od $k$. 
\end{definition}


\begin{definition}[Klasy W-hierarchii]
W-hierarchia to zbiór klas złożoności, oznaczonych jako $W[i]$.
Zparametryzowany problem znajduje się w klasie $W[i]$, jeśli można go zredukować do problemu \textit{Weighted Circuit Satisfiability} dla obwodów o głębokości ograniczonej do i \cite{parametrized_complexity_theory}. \\
Wierzy się że jeżeli problem jest $W[1]$-trudny, to nie należy do klasy FPT.
\label{def:whierarchy}
\end{definition}

\begin{definition}[Redukcja parametryczna]
Niech $(Q,\kappa)$ oraz $(Q',\kappa')$ będą zparametryzowanymi problemami decyzyjnymi, gdzie $\kappa$ i $\kappa'$ są funkcjami parametru.
Mówimy, że $(Q,\kappa)$ \emph{redukuje się parametrycznie} do $(Q',\kappa')$ (ozn. $(Q,\kappa)\leq_{\mathrm{fpt}}(Q',\kappa')$),
jeżeli istnieje funkcja obliczalna $R$ oraz funkcje obliczalne $f,g$, takie że dla każdego wejścia $x$ zachodzą warunki:
\begin{enumerate}
    \item $x \in Q \iff R(x) \in Q'$,
    \item $R(x)$ można obliczyć w czasie $f(\kappa(x)) \cdot |x|^{O(1)}$,
    \item $\kappa'(R(x)) \leq g(\kappa(x))$.
\end{enumerate}
\end{definition}

% \subsection{$W[1]$-trudność}
% \import{theory/fpt}{w1}

\subsection{$W[2]$-trudność}
\import{theory/fpt}{w2}

\subsection{Przynależność do $W[P]$}
\import{theory/fpt}{wp}
