Dowiedziona $\mathbb{NP}$-zupełność problemu naprawienia częściowego automatu dotyczy~przypadku, gdy z automatu usunięto wszystkie krawędzie. Jest to skrajna sytuacja. W~ramach tej pracy chcieliśmy się skupić na sytuacji, w której liczba brakujących krawędzi jest ograniczona. Zauważmy, że jeżeli w~automacie o $n$ stanach brakuje tylko jednej krawędzi, to do jego naprawy wystarczy sprawdzić $n$ automatów - po jednym dla każdego możliwego końca tej krawędzi. Tę obserwację można uogólnić do następującego faktu.

\begin{theorem}
Dla każdego ustalonego $k$, problem naprawiania częściowego automatu ograniczony
do automatów z co najwyżej $k$ brakującymi krawędziami (problem $k$-częściowego DFA) jest w~klasie problemów rozwiązywalnych w~czasie wielomianowym (klasa $\mathbb{P}$).
\label{th:ptime}
\end{theorem}

\begin{proof}
Niech dany będzie częściowy automat deterministyczny $A$ o $n$ stanach z co najwyżej $k$ brakującymi krawędziami. Algorytm polega na wyliczeniu wszystkich możliwych uzupełnień automatu poprzez dodanie brakujących krawędzi.

Każda brakująca krawędź może być skierowana do dowolnego spośród $n$ stanów, zatem liczba możliwych uzupełnień to co najwyżej $n^k$.

Dla każdego z tych uzupełnień sprawdzamy w~czasie wielomianowym (zależnym od rozmiaru automatu), czy powstały automat spełnia warunki specyfikacji. Ta weryfikacja wykonywana jest w~czasie $O(n^c)$ dla pewnej stałej $c$.

Całkowita złożoność wynosi $O(n^k \cdot n^c) = O(n^{k+c})$, co jest wielomianem względem rozmiaru wejścia dla ustalonego $k$. Zatem problem należy do $\mathbb{P}$ dla każdego ustalonego $k$. 

\end{proof}


Złożoność parametryczna (ang. \textit{parameterized complexity}) to framework do analizy problemów obliczeniowych, w~którym oprócz rozmiaru wejścia istnieje dodatkowy parametr. 
Zamiast klasyfikacji problemów jedynie jako należące do klasy $\mathbb{P}$ lub $\mathbb{NP}$-zupełnych, badamy zależność złożoności od tego parametru.

\vspace{0.5cm}

Z Twierdzenia \ref{th:ptime} można wywnioskować, że parametryczna wersja problemu naprawiania częściowego automatu należy do klasy XP, czyli klasy problemów, które można rozwiązać w~czasie $n^{f(k)}$, gdzie $f(k)$ jest funkcją tylko od parametru $k$. 

Nas interesuje dogłębniejsze zbadanie parametrycznej złożoności tego problemu, a~w~szczególności ustalenie, czy problem ten należy do klasy FPT, która jest parametrycznym odpowiednikiem klasy $\mathbb{P}$. 

Sparametryzowany problem to para $(Q, \kappa)$, gdzie $\kappa$ jest funkcją, która dla każdej instancji problemu ustala wartość parametru.

\begin{definition}[Klasa FPT]
% Klasa FPT składa się z problemów, dla których istnieje algorytm rozwiązujący je w~czasie $f(k) \cdot n^c$, gdzie $n$ to rozmiar wejścia, $f$ to dowolna funkcja obliczalna, a $c$ to stała niezależna od $k$. 

Klasa FPT (ang. \textit{Fixed-Parameter Tractable}) składa się z problemów, dla których istnieje algorytm, który dla wejścia $x$ działa w~czasie:
$$f(\kappa(x)) \cdot |x|^c$$
gdzie $|x|$ oznacza rozmiar wejścia, $f$ to dowolna funkcja obliczalna, a $c$ to stała niezależna od $x$.

Odpowiednikiem redukcji wielomianowych w~świecie złożoności parametrycznych są redukcje parametryczne.

\end{definition}


\begin{definition}[Redukcja parametryczna]
Niech $(Q,\kappa)$ oraz $(Q',\kappa')$ będą sparametryzowanymi problemami decyzyjnymi, gdzie $\kappa$ i~$\kappa'$ są funkcjami parametru.
Mówimy, że $(Q,\kappa)$ \textbf{redukuje się parametrycznie} do $(Q',\kappa')$ (ozn. $(Q,\kappa)\leq_{\mathrm{FPT}}(Q',\kappa')$),
jeżeli istnieje funkcja obliczalna $R$ (transformacja instancji) oraz funkcje obliczalne $f$ i $g$, takie że dla każdego wejścia $x$ zachodzą warunki:
\begin{enumerate}
    \item $x \in Q \iff R(x) \in Q'$,
    \item $R(x)$ można obliczyć w~czasie $f(\kappa(x)) \cdot |x|^{O(1)}$,
    \item $\kappa'(R(x)) \leq g(\kappa(x))$.
\end{enumerate}
\end{definition}

Dla porządku przywołamy tutaj definicję $W$-hierarchii.

\begin{definition}[Klasy $W$-hierarchii]
$W$-hierarchia to zbiór klas złożoności, oznaczonych jako $W[i]$ dla $i \geq 1$.
Sparametryzowany problem znajduje się w klasie $W[i]$, jeśli można go zredukować do problemu \textit{Weighted Circuit Satisfiability} dla obwodów o głębokości ograniczonej do $i$ \cite{parametrized_complexity_theory}. \\
Sparametryzowany problem jest $W[i]$-trudny, jeśli można zredukować problem \textit{Weighted Circuit Satisfiability} dla obwodów o głębokości ograniczonej do $i$, do tego problemu.

Dokładnej definicji problemu Weighted Circuit Satisfiability tu nie podajemy, gdyż nie będziemy z niej korzystać. Z definicji wprost wynika, że dla każdego $i$ zachodzi $W[i] \subseteq W[i+1]$. Zawierania w druga stronę są problemami otwartymi. Wierzy się, że jeżeli problem jest $W[1]$-trudny, to nie należy on do klasy FPT (analogicznie do relacji $\mathbb{P} \neq \mathbb{NP}$).
\label{def:whierarchy}
\end{definition}


\begin{definition}[Problem naprawiania $k$-częściowego DFA]
Problem naprawiania $k$-częściowego DFA to sparametryzowana wersja problemu naprawiania częściowego DFA, w której parametrem jest liczba brakujących krawędzi w automacie.

\noindent \textbf{Wejście:} parametr $k$, częściowy automat deterministyczny $A = (Q, \Sigma, \delta, q_\lambda, \mathbb{F_A}, \mathbb{F_R})$, w~którym dla dokładnie $k$ par $(q, a) \in Q \times \Sigma$ funkcja $\delta$ nie jest określona, zbiory próbek $S^+ \subseteq \Sigma^{*}$ i $S^{-} \subseteq \Sigma^{*}$.
\vspace{0.3cm}

\noindent \textbf{Wyjście:} odpowiedź, czy istnieje uzupełnienie brakujących przejść  tak, aby otrzymany automat był deterministyczny, akceptował wszystkie słowa z $S^+$ i odrzucał wszystkie słowa z $S^-$.

\end{definition}

% \subsection{$W[1]$-trudność}
% \import{theory/fpt}{w1}

\subsection{$W[2]$-trudność}
\import{theory/fpt}{w2}

\subsection{Przynależność do $W[P]$}
\import{theory/fpt}{wp}
