\section{Definicja problemu, framework}
% TODO
% definicje wstępne:
% - z jakich nazw zmiennych będziemy korzystać
% - czym jest DFA
% - czym są próbki

\begin{definition}[Deterministyczny automat skończony (DFA)]
Deterministyczny automat skończony (DFA) to szóstka uporządkowana $(Q, \Sigma, \delta, q_\lambda, \mathbb{F_A}, \mathbb{F_R})$, gdzie:
\begin{itemize}
    \item $Q$ to skończony zbiór stanów,
    \item $\Sigma$ to skończony alfabet wejściowy,
    \item $\delta: Q \times \Sigma \to Q$ to funkcja przejścia,
    \item $q_\lambda \in Q$ to stan początkowy,
    \item $\mathbb{F_A} \subseteq Q$ to zbiór stanów akceptujących,
    \item $\mathbb{F_R} \subseteq Q$ to zbiór stanów odrzucających.
\end{itemize}
\end{definition}


\subsection{Definicja problemu}

\begin{definition}[Problem naprawienia częściowego DFA]
    
    \textbf{Wejście:} częściowy automat deterministyczny $A = (Q, \Sigma, \delta, q_\lambda, \mathbb{F_A}, \mathbb{F_R})$, w którym:
\begin{itemize}
    \item niektóre krawędzie w automacie mogły zostać usunięte, więc dla niektórych par $(q, a) \in Q \times \Sigma$ funkcja $\delta$ nie jest określona,
    \item niektóre stany $q \in Q$ mogły zostać usunięte z automatu (brakujące stany), więc nie należą one ani do $\mathbb{F_A}$, ani do $\mathbb{F_R}$,
\end{itemize}
oraz zbiory próbek $S^+ \subseteq \Sigma^{*}$ (słowa akceptowane) i $S^{-} \subseteq \Sigma^{*}$ (słowa odrzucane).

\vspace{0.5cm}

\noindent \textbf{Wyjście:} odpowiedź, czy istnieje uzupełnienie brakujących przejść, klasyfikacji stanów i ewentualne dodanie stanów tak, aby otrzymany automat był deterministyczny, posiadał najmniejszą możliwą liczbę stanów oraz akceptował wszystkie słowa z $S^+$ i odrzucał wszystkie słowa z $S^-$. W przypadku istnienia, należy podać takie uzupełnienie.

\end{definition}


\subsection{Trywialność przypadku brakujących stanów}
W rozważanej wersji problemu dopuszczamy dodawanie brakujących stanów. Jest to jednak przypadek trywialny, bo możemy ograniczyć maksymalną liczbę stanów w automacie do liczby stanów w drzewie prefiksowym zbudowanym z próbek - jeżeli automat jest naprawialny, to będzie to automat rozwiązujący problem. Rozmiar takiego drzewa możemy ograniczyć przez $N = |S^+ \cup S^-| \cdot \max_{w \in S^+ \cup S^-} |w|$. Możemy więc dla każdej liczby stanów $n$ w zakresie $[1, N]$ sprawdzać, czy istnieje naprawa automatu z dokładnie $n$ stanami; od pewnej wartości $n$ odpowiedź staje się pozytywna, co pozwala użyć wyszukiwania binarnego bez istotnej zmiany (i tak wykładniczej) złożoności. 

Dlatego w dalszej części pracy zakładamy, że liczba stanów jest ustalona i nie rozważamy dodawania nowych.

\subsection{Definicja problemu (uproszczona)}
% TODO
% - wejście:
%     - DFA, w którym:
%         - niektóre przejścia są nieokreślone
%     - zbiór próbek pozytywnych $S^+$ (słowa akceptowane)
%     - zbiór próbek negatywnych $S^-$ (słowa odrzucane)
% - wyjście:
%     - czy istnieje sposób uzupełnienia brakujących przejść i stanów tak, aby otrzymany automat był deterministyczny oraz zgodny z dostarczonymi przykładami, tzn. akceptował wszystkie słowa z $S^+$ oraz odrzucał wszystkie słowa z $S^-$

% tą wersję problemu tak na prawdę będziemy analizować 

\begin{definition}[Problem naprawienia częściowego DFA (uproszczony)]
\textbf{Wejście:} częściowy automat deterministyczny $A = (Q, \Sigma, \delta, q_\lambda, \mathbb{F_A}, \mathbb{F_R})$, w którym dla pewnych par $(q, a) \in Q \times \Sigma$ funkcja $\delta$ nie jest określona, oraz zbiory próbek $S^+ \subseteq \Sigma^{*}$ i $S^{-} \subseteq \Sigma^{*}$. Liczba stanów $|Q|$ jest ustalona.
\vspace{0.3cm}

\noindent \textbf{Wyjście:} odpowiedź, czy istnieje uzupełnienie brakujących przejść i klasyfikacji stanów tak, aby otrzymany automat był deterministyczny, akceptował wszystkie słowa z $S^+$ i odrzucał wszystkie słowa z $S^-$. W przypadku istnienia należy podać takie uzupełnienie.
\end{definition}

\section{NP-zupełność}
% TODO
% Definicja NP zupełności


\subsection{Przynależność do NP}
% TODO


\subsection{NP-trudność}
% TODO
% Problem pasywnego uczenia to najogólniejszy przypadek naszego problemu (kiedy nie znamy żadnych przejść)



\section{FPT}
% TODO
\subsection{W[1]-trudność}
% TODO
% Redukcja k-klik do naszego problemu

\subsubsection{Dla $|\Sigma|=O(n)$}

\begin{figure}[h]
\centering
\begin{tikzpicture}[
        ->,
    >=Stealth,
    node distance=14mm,
    state/.style={circle, draw, minimum size=8mm},
    small/.style={circle, draw, minimum size=5mm},
    accept/.style={circle, draw, double, minimum size=8mm},
    reject/.style={circle, draw, minimum size=8mm},
    every label/.style={font=\small}
]
    % wierzchołek startowy
    \node[draw,state,inner sep=2pt] (s) at (0,-2) {$S$};

    % wierzchołek sink
    \node[draw,accept,inner sep=2pt] (sink) at (0,-4) {$ $};
    \draw[->,thick] (s) -- (sink) node[midway,above] {$t_1,t_2$};

    % wierzchołki v
    \node[draw,state,inner sep=1.5pt] (v1) at (4,0) {$v_{1}$};
    \node (vdots1) at (4,-0.7) {$\vdots$};
    \node[draw,state,inner sep=1.5pt] (vi) at (4,-1.6) {$v_{i}$};
    \node[draw,state,inner sep=1.5pt] (vj) at (4,-2.6) {$v_{j}$};
    \node (vdots2) at (4,-3.2) {$\vdots$};
    \node[draw,state,inner sep=1.5pt] (vn) at (4,-4.0) {$v_{n}$};

    % punkty docelowe dla krawędzi
    \coordinate (t1) at (3,-0.8);
    \coordinate (tx) at (3,-1.6);
    \coordinate (ty) at (3,-2.4);
    \coordinate (tlast) at (3,-3.2);

    % krawędzie p
    \draw[->,dashed,thick] (s) -- (t1) node[midway,above] {$p_{1}$};
    \node at (2.8,-1.2) {$\vdots$};
    \draw[->,dashed,thick] (s) -- (tx) node[midway,above] {$p_{x}$};
    \draw[->,dashed,thick] (s) -- (ty) node[midway,above] {$p_{y}$};
    \node at (2.8,-2.6) {$\vdots$};
    \draw[->,dashed,thick] (s) -- (tlast) node[midway,above] {$p_{k}$};

    % sekcja weryfikacji wierzchołka
    \node[draw,state,inner sep=2pt] (vv1) at (5.5,-0.2) {$S$};
    \node (vdots1) at (5.5,-0.8) {$\vdots$};
    \node[draw,state,inner sep=2pt] (vvi) at (5.5,-1.6) {$u_i$};
    \node (vdots1) at (5.5,-2.2) {$\vdots$};
    \node[draw,state,inner sep=2pt] (vvn) at (5.5,-3) {$S$};

    \draw[->,thick] (vi) -- (vv1) node[midway,above] {$1$};
    \draw[->,thick] (vi) -- (vvi) node[midway,above] {$i$};
    \draw[->,thick] (vi) -- (vvn) node[midway,above] {$n$};

    % sekcja wyboru
    \node[draw,state,inner sep=2pt] (choice1) at (8,-0.8) {$S$};
    \node[draw,state,inner sep=2pt] (choice2) at (8,-2.4) {$c_2$};
    \draw[->,thick] (vvi) -- (choice1) node[midway,above] {$\text{choice}_1$};
    \draw[->,thick] (vvi) -- (choice2) node[midway,above] {$\text{choice}_2$};

    % sekcja testu sąsiedztwa
    \node[accept, inner sep=1.5pt] (accept) at (11,-1.4) {};
    \node[reject, inner sep=1.5pt] (reject) at (11,-3) {};
    \draw[->,thick] (choice2) -- (accept) node[midway,above] {$v_j \in N(v_i)$};
    \draw[->,thick] (choice2) -- (reject) node[midway,above] {$v_j \notin N(v_i)$};

    % zielona ramka
    \node[
        draw=green!60!black,
        thick,
        rounded corners,
        fit=(vv1)(vvi)(vvn)(choice1)(choice2)(accept)(reject),
        label={[green!60!black]above:dla każdego $v_i$}
    ] {};

\end{tikzpicture}
\caption{Konstrukcja automatu dla redukcji z k-klik}
\label{fig:s-to-vi}
\end{figure}





\subsubsection{Dla $|\Sigma|=3$}


\subsubsection{Dla $|\Sigma|=2$}


\subsection{W[2]-trudność}
% TODO
% Redukcja z k-set cover (bierzemy k setów do pokrycia) do naszego problemu


\subsection{Przynależność do W[P]}
% TODO

% Póki co tylko nan bazie luźnej definicji z wikipedii - nasz prolem jest w klacie problemów gdzie mamy zbiór S n elementów (wszystkie możliwe naprawienia przejść), a my chcemy wybrać podzbiór z k - tak aby jakaś własność została utrzymana (być zgodnym z próbkami). Chcemy móc zapisać nasz wybór jako k intów zapisanych binarnie. \\
% Wg tej lużnej definicji, możemy mieć podejrzenia że nasz problem przynależy do W[P].

