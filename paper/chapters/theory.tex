W tym rozdziale przedstawimy dokładną definicję problemu naprawy częściowego DFA. Pokażemy, w jaki sposób problem można sprowadzić wyłącznie do brakujących krawędzi, wprowadzimy potrzebne oznaczenia oraz wykażemy $\mathbb{NP}$-zupełność tego problemu. Ponadto omówimy jego trudność w kontekście klas parametryzowanych, ze szczególnym uwzględnieniem hierarchii $W$ oraz klas $W[2]$ i $W[P]$.

\section{Definicja problemu, framework}

\begin{definition}[Deterministyczny automat skończony (DFA)]
Deterministyczny automat skończony (DFA) to krotka $(Q, \Sigma, \delta, q_\lambda, \mathbb{F_A}, \mathbb{F_R})$, gdzie:
\begin{itemize}
    \item $Q$ to skończony zbiór stanów,
    \item $\Sigma$ to skończony alfabet wejściowy,
    \item $\delta: Q \times \Sigma \to Q$ to funkcja przejścia,
    \item $q_\lambda \in Q$ to stan początkowy,
    \item $\mathbb{F_A} \subseteq Q$ to zbiór stanów akceptujących,
    \item $\mathbb{F_R} \subseteq Q$ to zbiór stanów odrzucających.
\end{itemize}
\end{definition}


\subsection{Definicja problemu z nieokreśloną liczbą stanów}

\begin{definition}[Problem naprawienia częściowego DFA z nieokreśloną liczbą stanów]
    
    \textbf{Wejście:} częściowy automat deterministyczny $A = (Q, \Sigma, \delta, q_\lambda, \mathbb{F_A}, \mathbb{F_R})$, w którym:
\begin{itemize}
    \item niektóre krawędzie w automacie mogły zostać usunięte, więc dla niektórych par $(q, a) \in Q \times \Sigma$ funkcja $\delta$ nie jest określona,
    \item niektóre stany mogły zostać usunięte z automatu (brakujące stany), więc nie należą one ani do $\mathbb{F_A}$, ani do $\mathbb{F_R}$,
\end{itemize}
oraz zbiory próbek $S^+ \subseteq \Sigma^{*}$ (słowa akceptowane) i $S^{-} \subseteq \Sigma^{*}$ (słowa odrzucane).

\vspace{0.5cm}

\noindent \textbf{Wyjście:} odpowiedź, czy istnieje uzupełnienie brakujących przejść, klasyfikacji stanów i ewentualne dodanie stanów tak, aby otrzymany automat był deterministyczny oraz akceptował wszystkie słowa z $S^+$ i odrzucał wszystkie słowa z $S^-$. W przypadku istnienia, należy podać takie uzupełnienie z najmniejszą liczbą dodatkowych stanów.

\end{definition}


\subsection{Redukcja przypadku brakujących stanów}
W rozważanej wersji problemu dopuszczamy dodawanie brakujących stanów. Możemy ograniczyć maksymalną liczbę stanów w automacie do liczby stanów w automacie, opartym na drzewie prefiksowym zbudowanym z próbek. Jeżeli automat jest naprawialny, to będzie to automat rozwiązujący problem. Rozmiar takiego drzewa możemy ograniczyć przez $N = |S^+ \cup S^-| \cdot \max_{w \in S^+ \cup S^-} |w|$. Możemy więc dla każdej liczby stanów $n$ w zakresie $[1, N]$ sprawdzać, czy istnieje naprawa automatu z dokładnie $n$ stanami; od pewnej wartości $n$ odpowiedź staje się pozytywna, co pozwala użyć wyszukiwania binarnego bez istotnej zmiany (i tak wykładniczej) złożoności. 
Tą weryfikację
Dlatego w dalszej części pracy zakładamy, że liczba stanów jest ustalona i nie rozważamy dodawania nowych.

\subsection{Definicja problemu z określoną liczbą stanów}

\begin{definition}[Problem naprawienia częściowego DFA]
\textbf{Wejście:} częściowy automat deterministyczny $A = (Q, \Sigma, \delta, q_\lambda, \mathbb{F_A}, \mathbb{F_R})$, w którym dla pewnych par $(q, a) \in Q \times \Sigma$ funkcja $\delta$ nie jest określona, oraz zbiory próbek $S^+ \subseteq \Sigma^{*}$ i $S^{-} \subseteq \Sigma^{*}$.
\vspace{0.3cm}

\noindent \textbf{Wyjście:} odpowiedź, czy istnieje uzupełnienie brakujących przejść i klasyfikacji stanów tak, aby otrzymany automat był deterministyczny, akceptował wszystkie słowa z $S^+$ i odrzucał wszystkie słowa z $S^-$. W przypadku istnienia należy podać takie uzupełnienie. \\

\noindent Zauważamy, że w tym problemie nie możemy dodawać nowych stanów do automatu.

\label{def:fixpartialdfa}
\end{definition}

\section{$\mathbb{NP}$-zupełność}
\import{theory}{npcomplex}

\section{Złożoność parametryzowana}
\import{theory}{fpt}
