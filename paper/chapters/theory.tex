\section{Definicja problemu, framework}
% TODO
% definicje wstępne:
% - z jakich nazw zmiennych będziemy korzystać
% - czym jest DFA
% - czym są próbki

\begin{definition}[Deterministyczny automat skończony (DFA)]
Deterministyczny automat skończony (DFA) to szóstka uporządkowana $(Q, \Sigma, \delta, q_\lambda, \mathbb{F_A}, \mathbb{F_R})$, gdzie:
\begin{itemize}
    \item $Q$ to skończony zbiór stanów,
    \item $\Sigma$ to skończony alfabet wejściowy,
    \item $\delta: Q \times \Sigma \to Q$ to funkcja przejścia,
    \item $q_\lambda \in Q$ to stan początkowy,
    \item $\mathbb{F_A} \subseteq Q$ to zbiór stanów akceptujących,
    \item $\mathbb{F_R} \subseteq Q$ to zbiór stanów odrzucających.
\end{itemize}
\end{definition}


\subsection{Definicja problemu}

\begin{definition}[Problem naprawienia częściowego DFA]
    
    \textbf{Wejście:} częściowy automat deterministyczny $A = (Q, \Sigma, \delta, q_\lambda, \mathbb{F_A}, \mathbb{F_R})$, w którym:
\begin{itemize}
    \item niektóre krawędzie w automacie mogły zostać usunięte, więc dla niektórych par $(q, a) \in Q \times \Sigma$ funkcja $\delta$ nie jest określona,
    \item niektóre stany $q \in Q$ mogły zostać usunięte z automatu (brakujące stany), więc nie należą one ani do $\mathbb{F_A}$, ani do $\mathbb{F_R}$,
\end{itemize}
oraz zbiory próbek $S^+ \subseteq \Sigma^{*}$ (słowa akceptowane) i $S^{-} \subseteq \Sigma^{*}$ (słowa odrzucane).

\vspace{0.5cm}

\noindent \textbf{Wyjście:} odpowiedź, czy istnieje uzupełnienie brakujących przejść, klasyfikacji stanów i ewentualne dodanie stanów tak, aby otrzymany automat był deterministyczny, posiadał najmniejszą możliwą liczbę stanów oraz akceptował wszystkie słowa z $S^+$ i odrzucał wszystkie słowa z $S^-$. W przypadku istnienia, należy podać takie uzupełnienie.

\end{definition}


\subsection{Trywialność przypadku brakujących stanów}
W rozważanej wersji problemu dopuszczamy dodawanie brakujących stanów. Jest to jednak przypadek trywialny, bo możemy ograniczyć maksymalną liczbę stanów w automacie do liczby stanów w drzewie prefiksowym zbudowanym z próbek - jeżeli automat jest naprawialny, to będzie to automat rozwiązujący problem. Rozmiar takiego drzewa możemy ograniczyć przez $N = |S^+ \cup S^-| \cdot \max_{w \in S^+ \cup S^-} |w|$. Możemy więc dla każdej liczby stanów $n$ w zakresie $[1, N]$ sprawdzać, czy istnieje naprawa automatu z dokładnie $n$ stanami; od pewnej wartości $n$ odpowiedź staje się pozytywna, co pozwala użyć wyszukiwania binarnego bez istotnej zmiany (i tak wykładniczej) złożoności. 

Dlatego w dalszej części pracy zakładamy, że liczba stanów jest ustalona i nie rozważamy dodawania nowych.

\subsection{Definicja problemu (uproszczona)}

\begin{definition}[Problem naprawienia częściowego DFA (uproszczony)]
\textbf{Wejście:} częściowy automat deterministyczny $A = (Q, \Sigma, \delta, q_\lambda, \mathbb{F_A}, \mathbb{F_R})$, w którym dla pewnych par $(q, a) \in Q \times \Sigma$ funkcja $\delta$ nie jest określona, oraz zbiory próbek $S^+ \subseteq \Sigma^{*}$ i $S^{-} \subseteq \Sigma^{*}$. Liczba stanów $|Q|$ jest ustalona.
\vspace{0.3cm}

\noindent \textbf{Wyjście:} odpowiedź, czy istnieje uzupełnienie brakujących przejść i klasyfikacji stanów tak, aby otrzymany automat był deterministyczny, akceptował wszystkie słowa z $S^+$ i odrzucał wszystkie słowa z $S^-$. W przypadku istnienia należy podać takie uzupełnienie.
\end{definition}

\section{NP-zupełność}
% TODO
% Definicja NP zupełności


\subsection{Przynależność do NP}
% TODO


\subsection{NP-trudność}
% TODO
% Problem pasywnego uczenia to najogólniejszy przypadek naszego problemu (kiedy nie znamy żadnych przejść)



\section{FPT}

\subsection{W[1]-trudność}

\begin{definition}[Problem klik]
\textbf{Wejście:} graf nieskierowany $G = (V, E)$.

\noindent \textbf{Wyjście:} klika w grafie $G$, czyli podzbiór wierzchołków $V' \subseteq V$ taki, że dla każdej pary wierzchołków $u, v \in V'$ zachodzi $(u, v) \in E$.

\noindent Problem kliki należy do problemów NP-zupełnych. 
\end{definition}


\begin{definition}[Problem k-klik]
\noindent Problem k-klik to zparametryzowana (\textit{Fixed Parameter Traceability}) wersja problemu kliki, w której dodatkowo podana jest liczba całkowita $k$ i należy odpowiedzieć, czy w grafie istnieje klika o rozmiarze co najmniej $k$.

\noindent\textbf{Wejście:} graf nieskierowany $G = (V, E)$ oraz liczba całkowita $k$.

\noindent \textbf{Wyjście:} odpowiedź, czy w grafie $G$ istnieje kliką o rozmiarze co najmniej $k$, czyli podzbiór $V' \subseteq V$ taki, że $|V'| \geq k$ oraz dla każdej pary wierzchołków $u, v \in V'$ zachodzi $(u, v) \in E$.

\noindent Problem k-klik należy do klasy W[1]-zupełnych problemów.
% TODO: link to source/reference
\end{definition}

Pokażemy, jak dla dowolnego wejścia do problemu k-klik skonstruować automat oraz próbki, będące wejściem do problemu naprawienia częściowego DFA, tak aby rozwiązanie problemu naprawienia częściowego DFA istniało wtedy i tylko wtedy, gdy w grafie istnieje klika o rozmżemy, jak dla dowolnego wejścia do problemu k-klik skonstruować automat oraz próbki, będące wejściem do problemu naprawienia częściowego DFA, tak aby rozwiązanie problemu naprawienia częściowego DFA istniało wtedy i tylko wtedy, gdy w grafie istnieje klika o rozmiarze co najmniej $k$. Dodatkowo pokażemy jak z rozwiązanego problemu naprawienia częściowego DFA wyprowadzić rozwiązanie problemu k-klik.


\subsubsection{Dla $|\Sigma|=O(n)$}
% TODO add complexity - time, memory, etc

Na rysunku \ref{fig:kcliquen} przedstawiona jest konstrukcja automatu dla redukcji z k-klik, korzystająca z alfabetu o rozmiarze zależnym od liczby wierzchołków grafu. \\
Każdy stan $q_{\lambda}$ odpowiada temu samemu stanowi początkowemu automatu.
Przejścia $p_i$ oznaczone przerywaną linią odpowiadają brakującym przejściom w automacie. Każde przejście, które nie jest zaznaczone, prowadzi do stanu odrzucającego (dowolne przejście ze stanu odrzucającego prowadzi z powrotem do niego).

\import{theory/figures}{reduction_kclique_n}

\noindent Idea konstrukcji: \\

Chcemy aby krawędzie $p_i$ prowadziły do stanów $v_j$, które reprezentują wierzchołki w grafie. Jeżeli krawędź $p_x$ prowadzi do stanu $v_i$, to oznacza, że wybieramy wierzchołek $v_i$ jako $x$-ty wierzchołek w klice. \\
Dodatkowo nie możemy wybrać tego samego wierzchołka wielokrotnie, więc musimy zagwarantować, że dla różnych przejść $p_x$ i $p_y$ wybieramy różne wierzchołki $v_i$ i $v_j$. \\
Na koniec musimy zagwarantować, że wybrane wierzchołki tworzą klikę, czyli że każdy wybrany wierzchołek jest połączony z każdym innym wybranym wierzchołkiem. \\

% TODO: remove figure V1 and replace it with figure V2 - then explain that edge 'test' is actually not neeeed
%* zagwarantowanie że wybieramy wierzchołki v_i
\textbf{Gwarancja prowadzenia brakujących krawędzi $p_x$ do wierzchołków $v_i$}

Zagwarantowanie tego faktu opiera się na prostej obserwacji: krawędź $\text{choice}_i$, nie prowadząca do stanu odrzucającego, znajduje się za wierzchołkami $v_i$. Oznacza to, że aby słowo zostało zaakceptowane, musimy przejść przez krawędź $p_x$ do któregoś ze stanów $v_i$, a następnie przez krawędź $\text{choice}_i$ do stanu akceptującego. \\
Aby uprościć tą obserwację, możemy dodać dodatkową krawędź między stanem $v_i$ a wierzchołkami $u_i$, tak jak na rysunku \ref{fig:kcliquen-v2}. \\
W takim przypadku, aby zagwarantować przejście krawędzią $p_x$ do któregoś ze stanów $v_i$, dodajemy w każdej próbce literę $\text{test}$ zaraz po literze $p_x$. W ten sposób, aby słowo zostało zaakceptowane, musimy przejść krawędzią $p_x$ do któregoś ze stanów $v_i$, a następnie krawędzią $\text{test}$ do stanu akceptującego. \\y

\import{theory/figures}{reduction_kclique_n_v2}

% TODO: zagwarantowanie że wybieramy różne wierzchołki dla różnych przejść p_i
\textbf{Wybór różnych wierzchołków $v_i$ dla różnych przejść $p_x$}
Aby uniemożliwić wybór tego samego $v_i$ dla różnych przejść $p_x$ oraz $p_y$ wystarczy stworzyć próbki postaci: 
$$\forall{x, y \in {1..k}, i \in {1..n}}   {s_{x, y, i} \in S^+  p_x \ \text{test} \ e_i \ \text{choice}_1 \ p_y \ \text{test} \ e_i }$$

% TODO explain why it works

% TODO: zagwarantowanie że wybrane wierzchołki tworzą klikę
\textbf{Gwarancja, że wybrane wierzchołki $v_i$ tworzą klikę w oryginalnym grafie}



\subsubsection{Dla $|\Sigma|=3$}

% TODO explain figure
\import{theory/figures}{reduction_kclique_3}


\subsubsection{Dla $|\Sigma|=2$}

% TODO explain figure
\import{theory/figures}{reduction_kclique_2}


\subsection{W[2]-trudność}
% TODO
% Redukcja z k-set cover (bierzemy k setów do pokrycia) do naszego problemu


\subsection{Przynależność do W[P]}
% TODO

% Póki co tylko nan bazie luźnej definicji z wikipedii - nasz prolem jest w klacie problemów gdzie mamy zbiór S n elementów (wszystkie możliwe naprawienia przejść), a my chcemy wybrać podzbiór z k - tak aby jakaś własność została utrzymana (być zgodnym z próbkami). Chcemy móc zapisać nasz wybór jako k intów zapisanych binarnie. \\
% Wg tej lużnej definicji, możemy mieć podejrzenia że nasz problem przynależy do W[P].

