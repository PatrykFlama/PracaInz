\section{Definicja problemu, framework}
% TODO
% definicje wstępne:
% - z jakich nazw zmiennych będziemy korzystać
% - czym jest DFA
% - czym są próbki

\begin{definition}[Deterministyczny automat skończony (DFA)]
Deterministyczny automat skończony (DFA) to szóstka uporządkowana $(Q, \Sigma, \delta, q_\lambda, \mathbb{F_A}, \mathbb{F_R})$, gdzie:
\begin{itemize}
    \item $Q$ to skończony zbiór stanów,
    \item $\Sigma$ to skończony alfabet wejściowy,
    \item $\delta: Q \times \Sigma \to Q$ to funkcja przejścia,
    \item $q_\lambda \in Q$ to stan początkowy,
    \item $\mathbb{F_A} \subseteq Q$ to zbiór stanów akceptujących,
    \item $\mathbb{F_R} \subseteq Q$ to zbiór stanów odrzucających.
\end{itemize}
\end{definition}


\subsection{Definicja problemu}

\begin{definition}[Problem naprawienia częściowego DFA]
    
    \textbf{Wejście:} częściowy automat deterministyczny $A = (Q, \Sigma, \delta, q_\lambda, \mathbb{F_A}, \mathbb{F_R})$, w którym:
\begin{itemize}
    \item dla niektórych par $(q, a) \in Q \times \Sigma$ funkcja $\delta$ nie jest określona,
    \item dla niektórych stanów $q \in Q$ nie jest określone, czy $q \in \mathbb{F_A}$ bądź $q \in \mathbb{F_R}$,
    \item niektóre stany mogą być całkowicie nieobecne (brakujące) i mogą zostać dodane,
\end{itemize}
oraz zbiory próbek $S^+ \subseteq \Sigma^{*}$ (słowa akceptowane) i $S^{-} \subseteq \Sigma^{*}$ (słowa odrzucane).

\vspace{0.5cm}

\noindent \textbf{Wyjście:} odpowiedź, czy istnieje uzupełnienie brakujących przejść, klasyfikacji stanów i ewentualne dodanie stanów tak, aby otrzymany automat był deterministyczny oraz akceptował wszystkie słowa z $S^+$ i odrzucał wszystkie słowa z $S^-$. W przypadku istnienia, należy podać jedno takie uzupełnienie.
% TODO możemy myśleć o wersji w której chcemy minimalizować liczbę dodanych stanów - jak się okaze nie ma to znaczenia

\end{definition}


\subsection{Trywialność przypadku brakujących stanów}
% TODO
% Wiemy ile może być max stanów (liczba próbek * długość max próbki) więc możemy albo to brutalnie przejść, co nie wpłynie znacznie na czas (mając wykładniczy składnik), albo nawet zbinsearchować (bo od pewnej liczby stanów zacznie istniej rozwiązanie/naprawialność).


\subsection{Definicja problemu (uproszczona)}
% TODO
% - wejście:
%     - DFA, w którym:
%         - niektóre przejścia są nieokreślone
%     - zbiór próbek pozytywnych $S^+$ (słowa akceptowane)
%     - zbiór próbek negatywnych $S^-$ (słowa odrzucane)
% - wyjście:
%     - czy istnieje sposób uzupełnienia brakujących przejść i stanów tak, aby otrzymany automat był deterministyczny oraz zgodny z dostarczonymi przykładami, tzn. akceptował wszystkie słowa z $S^+$ oraz odrzucał wszystkie słowa z $S^-$

% tą wersję problemu tak na prawdę będziemy analizować 

\section{NP-zupełność}
% TODO
% Definicja NP zupełności


\subsection{Przynależność do NP}
% TODO


\subsection{NP-trudność}
% TODO
% Problem pasywnego uczenia to najogólniejszy przypadek naszego problemu (kiedy nie znamy żadnych przejść)



\section{FPT}
% TODO
\subsection{W[1]-trudność}
% TODO
% Redukcja k-klik do naszego problemu


\subsection{W[2]-trudność}
% TODO
% Redukcja z k-set cover (bierzemy k setów do pokrycia) do naszego problemu


\subsection{Przynależność do W[P]}
% TODO

% Póki co tylko nan bazie luźnej definicji z wikipedii - nasz prolem jest w klacie problemów gdzie mamy zbiór S n elementów (wszystkie możliwe naprawienia przejść), a my chcemy wybrać podzbiór z k - tak aby jakaś własność została utrzymana (być zgodnym z próbkami). Chcemy móc zapisać nasz wybór jako k intów zapisanych binarnie. \\
% Wg tej lużniej definicji, możemy mieć podejrzenia że nasz problem przynależy do W[P].

