\section{Definicja problemu}
% TODO
% definicje wstępne:
% - z jakich nazw zmiennych będziemy korzystać
% - czym jest DFA
% - czym są próbki

\subsection{Definicja problemu 1}
% TODO
% - wejście:
%     - DFA, w którym:
%         - niektóre przejścia są nieokreślone - zwane później 'brakującymi przejściami'
%         - niektóre stany są nieokreślone (akceptujące/odrzucające)
%         - niektóre stany nie są podane (brakujące)
%     - zbiór próbek pozytywnych $S^+$ (słowa akceptowane)
%     - zbiór próbek negatywnych $S^-$ (słowa odrzucane)
% - wyjście:
%     - czy istnieje sposób uzupełnienia brakujących przejść i stanów tak, aby otrzymany automat był deterministyczny oraz zgodny z dostarczonymi przykładami, tzn. akceptował wszystkie słowa z $S^+$ oraz odrzucał wszystkie słowa z $S^-$ - jeżeli tak, to zwróć takie przykładowe uzupełnienie

% > możemy myśleć o wersji w której chcemy minimalizować liczbę dodanych stanów - jak się okaze nie ma to znaczenia


\subsection{Trywialność przypadku brakujących stanów}
% TODO
% Wiemy ile może być max stanów (liczba próbek * długość max próbki) więc możemy albo to brutalnie przejść, co nie wpłynie znacznie na czas (mając wykładniczy składnik), albo nawet zbinsearchować (bo od pewnej liczby stanów zacznie istniej rozwiązanie/naprawialność).


\subsection{Definicja problemu 2}
% TODO
% - wejście:
%     - DFA, w którym:
%         - niektóre przejścia są nieokreślone
%     - zbiór próbek pozytywnych $S^+$ (słowa akceptowane)
%     - zbiór próbek negatywnych $S^-$ (słowa odrzucane)
% - wyjście:
%     - czy istnieje sposób uzupełnienia brakujących przejść i stanów tak, aby otrzymany automat był deterministyczny oraz zgodny z dostarczonymi przykładami, tzn. akceptował wszystkie słowa z $S^+$ oraz odrzucał wszystkie słowa z $S^-$

% tą wersję problemu tak na prawdę będziemy analizować 

\section{NP-zupełność}
% TODO
% Definicja NP zupełności


\subsection{Przynależność do NP}
% TODO


\subsection{NP-trudność}
% TODO
% Problem pasywnego uczenia to najogólniejszy przypadek naszego problemu (kiedy nie znamy żadnych przejść)



\section{FPT}
% TODO
\subsection{W[1]-trudność}
% TODO
% Redukcja k-klik do naszego problemu


\subsection{W[2]-trudność}
% TODO
% Redukcja z k-set cover (bierzemy k setów do pokrycia) do naszego problemu


\subsection{Przynależność do W[P]}
% TODO

% Póki co tylko nan bazie luźnej definicji z wikipedii - nasz prolem jest w klacie problemów gdzie mamy zbiór S n elementów (wszystkie możliwe naprawienia przejść), a my chcemy wybrać podzbiór z k - tak aby jakaś własność została utrzymana (być zgodnym z próbkami). Chcemy móc zapisać nasz wybór jako k intów zapisanych binarnie. \\
% Wg tej lużniej definicji, możemy mieć podejrzenia że nasz problem przynależy do W[P].

