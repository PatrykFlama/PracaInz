\chapter*{Abstract}

\noindent 
Niniejsza praca dotyczy zagadnienia naprawy uszkodzonego deterministycznego automatu skończonego na podstawie próbek pozytywnych i negatywnych. Rozważany automat może zawierać brakujące krawędzie lub stany, co uniemożliwia jego poprawne działanie. 
W pracy analizowana jest złożoność obliczeniowa problemu, w tym jego przynależność do klasy $\mathbb{NP}$ oraz własności parametryzowane w kontekście algorytmów FPT i hierarchii $W$. 
Zaproponowano algorytm rozwiązujący rozważany problem, przeprowadzono jego analizę teoretyczną oraz przedstawiono implementację wraz z omówieniem optymalizacji i~heurystyk wpływających na czas działania.

\vspace{1cm}
\noindent\hrulefill 
\vspace{1cm}

\noindent 
This thesis addresses the problem of repairing a damaged deterministic finite automaton using positive and negative samples. The considered automaton may contain missing transitions or states, which prevents it from functioning correctly.
The work analyzes the computational complexity of the problem, including its membership in the class $\mathbb{NP}$ and its parameterized properties in the context of FPT algorithms and the $W$-hierarchy.
An algorithm for solving the problem is proposed, followed by its theoretical analysis, as well as an implementation accompanied by a discussion of optimizations and heuristics affecting the running time.


\iffalse
Opis pracy dyplomowej
Mamy dane dwa zbiory próbek: próbki pozytywne S+ i negatywne S- (to są bardzo duże zbiory słów), a dodatkowo deterministyczny automat, w którym usunięto/uszkodzono niektóre stany (lub niektóre przejścia). Naszym celem jest odbudowanie całego automatu tak, aby akceptował wszystkie słowa ze zbioru S+ i odrzucał wszystkie słowa ze zbioru S-.

Warianty problemu obejmują sytuację, gdy liczba brakujących stanów/krawędzi jest dowolna (dana na wejściu) oraz taką, gdy liczba ta jest ograniczona (np. w definicji problemu zakładamy, że rozważamy tylko automaty, którym brakuje co najwyżej 10 stanów).

Zgaduję, że przypadku, gdy liczba brakujących krawędzi jest dowolna, trzeba będzie pokazać NP-zupełność, a gdy ograniczona, problem jest w P (być może tylko dla ustalonego alfabetu). W takim razie praca obejmowałaby dowód takiej NP-zupełności, a także analizę wariantu ograniczonego, wraz z implementacją (być może z kilkoma heurystykami) i analizą wydajności.

Możliwe stopnie trudności obejmowałyby przeanalizowanie tego dla deterministycznych automatów skończonych, automatów z sumą (każda krawędź ma jakąś liczbę, a wartość słowa to suma wartości krawędzi) i ewentualnie automatów niedeterministycznych (ale to już chyba za trudne).
\fi

\thispagestyle{empty}
\pagebreak
