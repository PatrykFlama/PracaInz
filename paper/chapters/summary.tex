W niniejszej pracy wykazano, że rozważany problem jest W[1]-trudny oraz W[2]-trudny. Wyniki te pozwalają na umiejscowienie badanego zagadnienia w hierarchii klas złożoności parametryzowanej oraz wskazują na istotne ograniczenia możliwości skonstruowania algorytmów efektywnie parametryzowanych. Jednocześnie zasadne wydaje się podjęcie dalszych badań zmierzających do formalnego określenia, czy problem ten jest również W[P]-trudny, a także czy należy on do klasy W[P], co umożliwiłoby pełniejszą charakterystykę jego własności teoretycznych.

Przeprowadzone testy wskazują, że zaproponowany algorytm ze skokami stwarza potencjalne możliwości dalszej optymalizacji, w szczególności w zakresie zapotrzebowania na pamięć.

Istotnym kierunkiem dalszych badań jest również analiza wpływu struktury automatu na czas wykonania algorytmów. W szczególności interesujące wydaje się zbadanie automatu z różną ilością spójnych składowych.