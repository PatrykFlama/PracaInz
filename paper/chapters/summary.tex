W niniejszej pracy wykazano, że rozważany problem jest W[2]-trudny. Wyniki te pozwalają na umiejscowienie badanego zagadnienia w hierarchii klas złożoności parametrycznej oraz wskazują na istotne ograniczenia możliwości skonstruowania algorytmów efektywnie parametryzowanych. Jednocześnie zasadne wydaje się podjęcie dalszych badań zmierzających do formalnego określenia, czy problem ten jest również W[P]-trudny, a także czy należy on do klasy W[P], co umożliwiłoby pełniejszą charakterystykę jego własności teoretycznych.

Przeprowadzone testy wskazują, że zaproponowany algorytm ze skokami stwarza potencjalne możliwości dalszej optymalizacji, w szczególności w zakresie zapotrzebowania na pamięć.

Dodatkowym kierunkiem optymalizacji jest wykorzystanie struktury trie do wspólnego przetwarzania prefiksów próbek. Podejście to pozwala ograniczyć liczbę powtarzanych przejść automatu w przypadku zbiorów danych zawierających wiele próbek o wspólnych początkach. W praktyce może ono znacząco zmniejszyć liczbę operacji symulacji, zwłaszcza dla dużych zbiorów uczących.

Obiecującym rozszerzeniem algorytmu pełnego przeszukiwania jest również zastosowanie backtrackingu w połączeniu z mechanizmem skoków. Takie podejście umożliwia wczesne odrzucanie nieperspektywicznych częściowych konfiguracji automatu oraz kontynuowanie symulacji próbek od ostatniego poprawnie zdefiniowanego przejścia. W efekcie zmniejsza to zarówno liczbę analizowanych kombinacji uzupełnień, jak i koszt ich weryfikacji.

Istotnym kierunkiem dalszych badań jest również analiza wpływu struktury automatu na czas wykonania algorytmów. W szczególności interesujące wydaje się zbadanie automatu z różną ilością spójnych składowych.